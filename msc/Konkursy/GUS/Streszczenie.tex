\documentclass[]{article}

\usepackage{polski}
\usepackage[polish]{babel}
\usepackage{amsfonts}
\usepackage{indentfirst}
\usepackage[utf8]{inputenc}
\usepackage{amsthm}
\usepackage{multirow}
\usepackage{amsmath}

\begin{document}
\begin{center}
	\huge{{\bf Streszczenie}}\\ Grzegorz Mika\\Nierówności wyrocznie dla problemów odwrotnych
\end{center}
\vspace{1cm}
W pracy rozważany jest problem estymacji nieznanego elementu $f$ na podstawie niebezpośrednich i zaburzonych obserwacji w modelu 
\begin{displaymath}
Y = Af + \epsilon\xi.
\end{displaymath}. Niech $\Lambda$ będzie skończonym zbiorem estymatorów liniowych, czyli estymatorów postaci $TY$, gdzie $T$ jest pewnym operatorem liniowym. Celem jest konstrukcja metody wyboru estymatora z rodziny $\Lambda$ naśladującego estymator o minimalnym ryzyku w tej klasie. Okazuje się, że można to osiągnąć poprzez minimalizację odpowiedniego wyrażenia związanego z ryzykiem empirycznym, nieobciążonego estymatora tego ryzyka w przypadku problemu z operatorem zwartym bądź górnego oszacowania tego ryzyka w problemie z dowolnym operatorem liniowym ciągłym. Głównym wynikiem pracy jest zaprezentowanie odpowiednich nieasymptotycznych nierówności wyroczni w obu przypadkach postaci 
\begin{displaymath}
\mathcal{R}(\theta^*,\theta)\leq C_1 \inf_{\hat{\theta} \in \Lambda}\mathcal{R}(\hat{\theta},\theta)+C_2(\Lambda,n),
\end{displaymath}
gdzie $\Lambda$ jest rozważaną rodziną estymatorów, $\mathcal{R}$ jest odpowiednio zdefiniowanym ryzykiem estymatora, $\theta$ poszukiwanym elementem a estymator $\theta^*$ jest wybierany zgodnie z zaprezentowaną metodą. \\
Dodatkowo pokazane zostaje w pracy, że przy pewnych założeniach na rozważaną rodzinę estymatorów $\Lambda$, powyższe nierówności wyrocznie są asymptotycznie dokładne, czyli
\begin{equation}\label{aoracle}
\mathcal{R}(\theta^*,\theta)\leq(1+o(1))\inf_{\hat{\theta} \in \Lambda}\mathcal{R}(\hat{\theta},\theta).
\end{equation}

\end{document}
