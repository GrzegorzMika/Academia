\documentclass{beamer}
\usepackage[polish]{babel}
\usepackage[utf8]{inputenc}
\usepackage{lmodern}
\usepackage{polski}
\usepackage{amsfonts}
\usepackage{eufrak}
\usepackage{indentfirst}
\usepackage{amsthm}
\usepackage{multirow}
\usepackage{amsmath}
\usepackage{graphicx}
\newcommand{\norm}[1]{\left\lVert#1\right\rVert}
\usetheme{AGH}

\title{Nierówności wyrocznie dla problemów odwrotnych}
\subtitle{Praca magisterska}
\author{Grzegorz Mika}
\institute{Wydział Matematyki Stosowanej}

\date{5 września 2018}

\begin{document}

\titleframe[pl]

\begin{frame}\frametitle{Plan prezentacji}
\begin{itemize}
\item Przedstawienie problemu 
\vspace{3mm}
\item Cel pracy
\vspace{3mm}
\item Nierówności wyrocznie w modelu z operatorem zwartym
\vspace{3mm}
\item Nierówności wyrocznie w ogólnym modelu
\end{itemize}
\end{frame}

\begin{frame}\frametitle{Przedstawienie problemu}
\begin{block}{Problem}
Niech $A\colon \mathbb{H}\to \mathbb{G}$ będzie operatorem liniowym i ograniczonym między ośrodkowymi przestrzeniami Hilberta. Rozważmy problem
\begin{displaymath}
Af = g.
\end{displaymath}
Cel: mając dany $g \in \mathbb{G}$ znajdź $f \in \mathbb{H}$  by spełnione było powyższe równanie.
\end{block}
\begin{block}{Stochastyczny problem odwrotny}
Równanie zaburzone przez losowy szum
\begin{displaymath}
Y = Af + \epsilon\xi,\ \epsilon > 0.
\end{displaymath}
\end{block}
\end{frame}

\begin{frame}\frametitle{Przedstawienie problemu}
\begin{block}{Dobrze i źle postawione problemy}
Problem nazwiemy dobrze postawionym w sensie Hadamarda, gdy:
\begin{itemize}
\item dla dowolnego $g\in G$ istnieje $f\in H$ spełniający zadane równanie,
\item rozwiązanie jest jedyne,
\item rozwiązanie jest stabilne, czyli zależy w sposób ciągły od prawej strony równania.
\end{itemize}
Problem nazwiemy źle postawionym, gdy nie jest dobrze postawiony.
\end{block}
\end{frame}

\begin{frame}\frametitle{Błąd stochastyczny}
\begin{block}{Błąd stochastyczny}
Stochastycznym błędem $\xi$ nazwiemy proces na przestrzeni Hilberta G, czyli ograniczony liniowy operator $\xi\colon G\to L_2(\Omega, \mathcal{F},\mathbb{P})$ taki, że dla dowolnych elementów $g_1,g_2\in G$ mamy zdefiniowane zmienne losowe $\langle \xi, g_i\rangle$ takie, że $\mathbb{E}\langle \xi, g_i\rangle =0$. Operator kowariancji $\textbf{Cov}_{\xi}$ określony jest jako ograniczony liniowy operator ($||\textbf{Cov}_{\xi}||\leq 1$) z przestrzeni $G$ w przestrzeń $G$ taki, że $ \langle \textbf{Cov}_{\xi}g_1,g_2\rangle=\textbf{Cov}(\langle \xi,g_1\rangle,\langle \xi,g_2\rangle)$. 
\end{block}
\end{frame}


\begin{frame}\frametitle{Błąd stochastyczny}
\begin{block}{Działania na szumie}
Niech $T$ będzie operatorem na przestrzeni $H$. Wtedy można zdefiniować szum $T\xi$ przez
\begin{displaymath}
\langle T\xi , f\rangle=\langle \xi, T^* f\rangle,\ \forall f\in TH.
\end{displaymath}
Operator kowariancji $T\xi$ jest postaci $\textbf{Cov}_{T\xi}=T\textbf{Cov}_{\xi}T^*$.
\end{block}
\begin{block}{Estymator liniowy}
Estymatorem liniowym elementu $f$ w powyższym modelu nazywamy estymator postaci $TY$, gdzie $T$ jest pewnym operatorem z przestrzeni $L(G,H)$. 
\end{block}
\end{frame}


\begin{frame}\frametitle{Cel pracy}
\begin{center}
Niech $\Lambda$ będzie skończonym zbiorem estymatorów liniowych.
\end{center}
\begin{block}{Cel}
Wybór z rodziny $\Lambda$ estymatora naśladującego estymator o minimalnym ryzyku w tej klasie.
\end{block}
\begin{block}{Nierówności wyrocznie}
Niech $\mathcal{R}(\theta^*,\theta)$ oznacza ryzyko estymatora $\theta^*$. Będziemy poszukiwać metody wyboru $\theta^*$ z $\Lambda$ tak, by udało się skonstruować nierówność następującej postaci
\begin{displaymath}
\mathcal{R}(\theta^*,\theta)\leq C_1 \inf_{\hat{\theta} \in \Lambda}\mathcal{R}(\hat{\theta},\theta)+C_2(\Lambda,n).
\end{displaymath}
\end{block}
\end{frame}


\begin{frame}\frametitle{Model z operatorem zwartym}
\begin{center}
Niech w problemie $Y=Af+\epsilon\xi$ $A$ będzie operatorem zwartym.
\end{center}
\begin{block}{Uwaga}
Jeżeli $A\colon \mathbb{H}\to \mathbb{G}$ jest zwarty i $dim \mathbb{H} = \infty$, to $A^{-1}$ jest nieciągły, o ile istnieje, czyli problem $Af = g$ jest źle postawiony dla dowolnego operatora zwartego.
\end{block}
\end{frame}

\begin{frame}\frametitle{Model z operatorem zwartym}
\begin{block}{Reprezentacja według wartości singularnych}
Istnieją skończony lub zbieżny do zera ciąg liczb dodatnich $\{b_n\}_{n\in I}$ oraz układy ortonormalne $\{v_n\}_{n\in I}\subset H,\ \{u_n\}_{n\in I}\subset G$ takie, że zachodzi
\begin{itemize}
\item $KerA^{\perp}=\overline{span\{v_n,\ n\in I\}}$, $\overline{RangeA}=\overline{span\{u_n,\ n\in I\}}$,
\item $Af=\sum_nb_n\langle f, v_n\rangle u_n$ oraz $A^*g=\sum_nb_n\langle g, u_n\rangle v_n$.
\end{itemize}
Przy pewnych dodatkowych warunkach rozwiązania równania $Af=g$ mają postać 
\begin{displaymath}
f=f_0+\sum_nb_n^{-1}\langle g, u_n\rangle v_n,\ f_0\in KerA.
\end{displaymath}
\end{block}
\end{frame}

\begin{frame}\frametitle{Model z operatorem zwartym}
\begin{block}{Model ciągowy}
Problem $Y = Af +\epsilon\xi$ ma reprezentację postaci
\begin{displaymath}
x_n=\theta_n+\epsilon\sigma_n\xi_n,\ n=1,2,\dots
\end{displaymath}
gdzie $x_n=y_n/b_n = \langle Y, u_n\rangle /b_n$, $\theta_n = \langle f, v_n\rangle$ oraz $\sigma_n=b_n^{-1}$.
\end{block}
\begin{block}{Estymator liniowy}
W powyższym modelu estymator liniowy ma postać $\hat{\theta}(\lambda)=(\hat{\theta}_1,\hat{\theta}_2,\dots)$, gdzie
\begin{displaymath}
\hat{\theta}_i=\lambda_ix_i,\ i=1,2,\dots
\end{displaymath}
dla pewnego nielosowego ciągu liczbowego $\lambda=(\lambda_1,\lambda_2,\dots)$.
\end{block}
\end{frame}

\begin{frame}\frametitle{Model z operatorem zwartym}
\begin{block}{Ryzyko}
Ryzyko estymatora $\theta^*$ w powyższym modelu wyraża się jako 
\begin{displaymath}
\begin{split}
\mathcal{R}(\hat{f},f)& = \mathbb{E}_{\theta}||\theta-\hat{\theta}||^2  \\
&=\sum_{n=1}^{\infty}(1-\lambda_n)^2\theta_n^2+\epsilon^2\sum_{n=1}^{\infty}\sigma_n^2\lambda_n^2.
\end{split}
\end{displaymath}
\end{block}
\begin{block}{Nieobciążony estymator ryzyka}
Nieobciążonym estymatorem $\mathcal{R}(\hat{\theta},\theta)-\sum_{n=1}^{\infty}\theta_n^2$ nazwiemy wyrażenie
\begin{displaymath}
U(\lambda,X)=\sum_{n=1}^{\infty}(\lambda_n^2-2\lambda_n)x_n^2+2\epsilon^2\sum_{n=1}^{\infty}\lambda_n\sigma_n^2
\end{displaymath}
\end{block}
\end{frame}


\begin{frame}\frametitle{Model z operatorem zwartym}
\begin{block}{Oznaczenia}
\begin{displaymath}
\begin{split}
&\rho=\max_{\lambda\in \Lambda}\sup_n\sigma_n^2|\lambda_n|\left[\sum_{k=1}^{\infty}\sigma_k^4\lambda_k^4\right]^{-1/2},\\
&S=\frac{\max_{\lambda\in\Lambda}\sup_n\sigma_n^2\lambda_n^2}{\min_{\lambda\in \Lambda}\sup_n\sigma_n^2\lambda_n^2},\ M=\sum_{\lambda\in \Lambda}\exp\left(\frac{-1}{\rho(\lambda)}\right),\\
&L_{\lambda}=\ln(NS)+\rho^2\ln^2(MS).
\end{split}
\end{displaymath}
\end{block}
\begin{block}{Metoda}
\begin{displaymath}
\lambda^*=\arg\min_{\lambda\in \Lambda}U(\lambda,X)
\end{displaymath}
\end{block}
\end{frame}


\begin{frame}\frametitle{Model z operatorem zwartym}
\begin{center}
Przy odpowiednich założeniach zachodzą następujące nierówności dla estymatora $\theta^*$ wybranego zaproponowaną metodą.
\end{center}
\begin{block}{Wyniki}
Dla dowolnego $\theta\in l^2$ i dla dowolnego $B>B_0$ istnieją stałe $\gamma_1,\gamma_2,\gamma_3,\gamma_4$ takie, że zachodzi
\begin{displaymath}
\begin{split}
\bullet\ \mathbb{E}_{\theta}\norm{\theta^*-\theta}^2&\leq (1+\gamma_1B^{-1})\min_{\lambda\in \Lambda}\mathcal{R}(\hat{\theta},\theta)+\gamma_2B\epsilon^2L_{\Lambda}\omega(B^2L_{\Lambda})\\
\bullet\ \mathbb{E}_{\theta}\norm{\theta^*-\theta}^2&\leq (1+\gamma_3\rho\sqrt{L_{\Lambda}})\min_{\lambda\in \Lambda}\mathcal{R}(\hat{\theta},\theta),
\end{split}
\end{displaymath}
o ile $\rho\sqrt{L_{\Lambda}}<\gamma_4$. Funkcja $\omega(x)$ jest postaci
\begin{center}
$\omega(x)=\max_{\lambda\in \Lambda}\sup_k\left[\sigma_k^2\lambda_k^2\pmb{1}\left(\sum_{n=1}^{\infty}\sigma_n^2\lambda_n^2\leq x \sup_k\sigma_k^2\lambda_k^2\right)\right],\ x>0.$
\end{center}
\end{block}
\end{frame}



\begin{frame}\frametitle{Model z dowolnym operatorem}
\begin{center}
Niech w problemie $Y=Af+\epsilon\xi$ $A$ będzie dowolnym operatorem liniowym i ograniczonym.
\end{center}
\begin{block}{Prekondycjonowanie problemu}
W miejsce problemu 
\begin{displaymath}
Y=Af+\epsilon\xi
\end{displaymath}
rozważać będziemy problem
\begin{displaymath}
A^*Y=A^*Af+\epsilon A^*\xi.
\end{displaymath}
\end{block}
\end{frame}

\begin{frame}\frametitle{Model z dowolnym operatorem}
\begin{block}{Twierdzenie spektralne}
Niech $A$ będzie operatorem samosprzężonym na ośrodkowej przestrzeni Hilberta $H$. Wtedy istnieją $\sigma$-- zwarta przestrzeń mierzalna $(S,\mathcal{S},\mu )$ z miarą Radona $\mu$, rzeczywista funkcja mierzalna $b$ określona na $S$ i operator unitarny $U\colon H\to L_2(S,\mathcal{S},\mu )$, takie, że 
\begin{displaymath}
A=U^{-1}M_bU,
\end{displaymath}
gdzie $M_b$ jest operatorem mnożenia przez funkcję $b$ zdefiniowanym jako $(M_bg)(x)=b(x)g(x)$
\end{block}
\end{frame}

\begin{frame}\frametitle{Model z dowolnym operatorem}
\begin{block}{Postać}
Problem $Y=Af+\epsilon\xi$ można zapisać w postaci
\begin{displaymath}
X=\theta +\epsilon\sigma\eta,
\end{displaymath}
gdzie $\theta =Uf$, $\sigma\eta = M_{b^{-1}}UA^*\xi$ oraz $X = M_{b^{-1}}UA^*Y$.
\end{block}
\begin{block}{Estymator}
W powyższym modelu estymator liniowy $\hat{\theta}$ przyjmuje postać 
\begin{displaymath}
\hat{\theta}=\lambda X,
\end{displaymath}
gdzie $\lambda$ jest pewną nielosową funkcją. 
\end{block}
\end{frame}


\begin{frame}\frametitle{Model z dowolnym operatorem}
\begin{block}{Założenia na szum}
Załóżmy, że zachodzą następujące warunki gwarantujące skończoność drugich momentów regularyzowanego rozwiązania
\begin{displaymath}
\forall g\in G\hspace{0.5cm}  \mathbb{E}\langle \xi, g\rangle =0\textrm{ oraz }\norm{\textbf{Cov}_{\xi}}\leq 1,
\end{displaymath}
\begin{displaymath}
\mathbb{E}\norm{A^*\xi}^2<\infty.
\end{displaymath}
Jeżeli spełnione są powyższe dwa warunki, możliwe jest takie dobranie miary $\mu$ i  operatora $U$ w twierdzeniu spektralnym, by zachodziło
\begin{displaymath}
\forall s\in S\hspace{0.5cm} \textbf{Var}(UA^*\xi (s))\leq b(s)
\end{displaymath}
\end{block}
\end{frame}


\begin{frame}\frametitle{Model z dowolnym operatorem}
\begin{block}{Oszacowanie na ryzyko}
Przy założeniach na szum ryzyko estymatora $\hat{\theta}$ można oszacować 
\begin{displaymath}
\mathcal{R}(\hat{\theta},\theta ) \leq \int_S(1-\lambda(s))^2\theta^2(s)d\mu+\epsilon^2\int_S\lambda^2(s)\sigma(s)d\mu.
\end{displaymath}
\end{block}
\begin{block}{Oznaczenie}
Wprowadźmy oznaczenie 
\begin{displaymath}
\Psi(\lambda,\theta) = \int_S(1-\lambda(s))^2\theta^2(s)d\mu+\epsilon^2\int_S\lambda^2(s)\sigma(s)d\mu.
\end{displaymath}
\end{block}
\end{frame}

\begin{frame}\frametitle{Model z dowolnym operatorem}
\begin{block}{Empiryczne oszacowanie ryzyka}
Wprowadźmy następującą wielkość
\begin{displaymath}
\psi(\lambda,X)=\int_S(\lambda^2-2\lambda)X^2 d\mu+ 2\epsilon^2\int_S\lambda\sigma d\mu.
\end{displaymath}
Zachodzi następująca zależność
\begin{displaymath}
\mathbb{E}\psi (\lambda,X)\leq \Psi(\lambda,\theta)-\int_S\theta^2d\mu.
\end{displaymath}
\end{block}
\end{frame}


\begin{frame}\frametitle{Model z dowolnym operatorem}
\begin{block}{Metoda}
Poszukiwany estymator $\theta^*$ będzie konstruowany tak, by filtr był wybierany z rodziny $\Lambda$ zgodnie z zasadą
\begin{displaymath}
\lambda^*=\arg\min_{\lambda\in \Lambda}\psi(\lambda,X)
\end{displaymath}
\end{block}
\end{frame}

\begin{frame}\frametitle{Model z dowolnym operatorem}
\begin{center}
\tiny{
Przy odpowiednich założeniach, oznaczeniach analogicznych do modelu z operatorem zwartym i estymatorze $\theta^*$ wybranym zgodnie z zaproponowaną metodą można pokazać następujące nierówności}
\end{center}
\begin{block}{Wyniki}
\scriptsize{Dla dowolnego $\theta\in L_2(S,\mathcal{S},\mu)$, dla dowolnego $B>B_0$ istnieją stałe $\gamma_1, \gamma_2,\gamma_3, \gamma_4$ takie, że zachodzi}
\small{\begin{displaymath}
\begin{split}
\bullet\ \mathbb{E}_{\theta}&\norm{\theta^*-\theta}^2 \leq \max \{1,C_2\}(1+\gamma_1B^{-1}\max\{1,\norm{A}^2\})\min_{\lambda\in \Lambda}\Psi(\lambda,\theta)\\
 &\hspace{21mm} + \max \{1,C_2\}\gamma_2B\max\{1,\norm{A}^2\}\epsilon^2L_{\Lambda}\omega(B^2L_{\Lambda})\\
\bullet\ \mathbb{E}_{\theta}&\norm{\theta^*-\theta}^2\leq \max \{1,C_2\}(1+\gamma_3\max\{1,\norm{A}^2\}\sqrt{ \rho L_{\Lambda}})\min_{\lambda\in \Lambda}\Psi(\lambda,\theta),
\end{split}
\end{displaymath}}
\scriptsize{o ile $\rho\sqrt{L_{\Lambda}}<\gamma_4$. Funkcja $\omega(x)$ jest postaci}
\small{\begin{center}
$
\omega(x)=\max_{\lambda\in \Lambda}\norm{\sigma^2\lambda^2\pmb{1}\left(\int_S\sigma\lambda^2d\mu\leq x \norm{\sigma^2\lambda^2}_{\infty}\right)}_{\infty},\ x>0.$
\end{center}}
\end{block}
\end{frame}




\begin{frame}\frametitle{Koniec}
\begin{center}
\huge{\textbf{Dziękuję za uwagę!}}
\end{center}
\end{frame}
\end{document}
