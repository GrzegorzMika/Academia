\documentclass[woman,mfiu]{mgrwms}
\usepackage{polski}
\usepackage[utf8]{inputenc}    
\usepackage{amsthm}
\usepackage{amsmath}           
\usepackage{amssymb}
\usepackage{amsfonts}
\usepackage{enumerate}
\usepackage[T1]{fontenc}  
\usepackage{lmodern}
\usepackage{fancyhdr}
\usepackage{indentfirst}
\usepackage{latexsym}
\usepackage{graphicx}
\usepackage{bbm}
\usepackage{dsfont}
\usepackage{mathrsfs}
\usepackage{array}
\usepackage{eqnarray}
%\usepackage{hyperref}  %odsyłacze
\usepackage{multicol}
\usepackage{url}

\usepackage{listings}
 
\usepackage{xcolor}
\usepackage{epstopdf}
\usepackage{sidecap}
\usepackage{wrapfig}
\usepackage{subfig}
\usepackage{float}
\usepackage{bbm}
%\usepackage[a4paper, left=3.2cm, right=2.5cm, top=1.75cm, bottom=2.2cm]{geometry}
\newcommand{\slfrac}[2]{\left.#1\middle/#2\right.}




\usepackage{hyperref} % pakiet pozwalający na odwołania w postaci linków

\usepackage{cleveref}
%\usepackage{amsmath,times,latexsym}           % łatwiejszy skład matematyki

\usepackage{multicol}

\usepackage{multirow}
\usepackage{pgf,tikz}
\usepackage{mathrsfs}
\usepackage{extarrows}
\usepackage{hyphenat} % dzielenie wyrazów
%\hypersetup{pdfpagemode=FullScreen} % wyświetla pdf w pełnym ekranie





\lstset{
	%basicstyle=\footnotesize,      
 	numbers=left,                   
  	numberstyle=\tiny\color{gray},  
 	stepnumber=1,                  
    numbersep=10pt,                 
 	frame=single,  
 	lineskip=2pt            
} 
\DeclareMathOperator{\sgn}{sgn}
\renewcommand\arraystretch{1.5}


\begin{document}
{\bf{\centerline{\begin{tabular}{c}Akademia Górniczo-Hutnicza
im. Stanisława Staszica w Krakowie \\ Wydział Matematyki Stosowanej
\\\hline \end{tabular}}}}
\vskip0.5cm \rightline{\it{Kraków, \today}}
\leftline{\begin{tabular}{c}prof. dr hab. inż. Zbigniew Szkutnik\\ \hline
\small{\it{(opiekun)}}
\end{tabular}} \vskip0.5cm
\begin{center}
\begin{tabular}{l}
\Large{OCENA PRACY MAGISTERSKIEJ}
\end{tabular}
\end{center}
 \vskip0.5cm
\rm {\bf{
Temat pracy: Nierówności wyrocznie dla problemów odwrotnych \\
\\Imię i nazwisko: Grzegorz Mika \hskip4cm Nr albumu: 267543
}}
\begin{enumerate}
\item {\bf{Czy treści pracy odpowiadają tematowi określonemu w
tytule}}:  (do uzupełnienia)
\\
\item {\bf{Merytoryczna ocena pracy}}: (do uzupełnienia)
\\
\item {\bf{Czy i w jakim zakresie praca stanowi nowe ujęcie problemu}}: (do uzupełnienia)
\\
\item {\bf{Ocena formalnej strony pracy}}: (do uzupełnienia)
\\
\item {\bf{Końcowa ocena pracy}}: (do uzupełnienia)
\end{enumerate}
\end{document}