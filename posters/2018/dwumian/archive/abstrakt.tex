\documentclass{article}
\usepackage{polski}
\usepackage[polish]{babel}
\usepackage{amsfonts}
\usepackage{eufrak}
\usepackage{indentfirst}
\usepackage[utf8]{inputenc}
\usepackage{amsthm}
\usepackage{multirow}
\usepackage{amsmath}
\usepackage{makeidx}
\newtheorem{tw}{Twierdzenie}
\newtheorem{df}{Definicja}
\newtheorem{lm}{Lemat}
\newtheorem*{lem}{Lemat}
\newtheorem{wn}{Wniosek}
\newtheorem{prz}{Przykład}
\newtheorem{uw}{Uwaga}
\newtheorem{za}{Założenie}
\newcommand{\norm}[1]{\left\lVert#1\right\rVert}
\title{Dekonwolucja zaburzonego sygnału.\\Podejście przez wyrocznie.}
\author{Grzegorz Mika}
\begin{document}
\maketitle
\begin{abstract}
Rozważony zostanie  problem estymacji nieznanego elementu $f$ na podstawie niebezpośrednich i zaburzonych obserwacji. Zaburzenie modelowane będzie przez pewien zwarty operator liniowy $A$. Niech $\Lambda$ będzie pewnym skończonym zbiorem estymatorów liniowych. Celem będzie konstrukcja metody wyboru estymatora z rodziny $\Lambda$ naśladującego estymator o minimalnym ryzyku w tej klasie. Okaże się, że można to osiągnąć poprzez minimalizację nieobciążonego estymatora ryzyka. W drugiej części zaprezentowany zostanie numeryczny przykład porównujący tą metodą z popularną metodą rozbieżności.
\end{abstract}
\end{document}