\documentclass[12pt]{article}
\usepackage{polski}
\usepackage[polish]{babel}
\usepackage{amsfonts}
\usepackage{indentfirst}
\usepackage[utf8]{inputenc}
\usepackage{amsthm}
\usepackage{multirow}
\usepackage{amsmath}
\newtheorem{tw}{Twierdzenie}
\newtheorem{df}{Definicja}
\newtheorem{zd}{Zadanie}
\newtheorem{prz}{Przykład}
\title{Operatory domknięte i domykalne}
\begin{document}
\maketitle
W pracy tej zebrano i podsumowane pewne pojęcia, twierdzenia i własności związane z operatorami domkniętym i domykalnymi.\par


Rozważmy dwie przestrzenie Banacha $X,Y$ i niech $A\colon X \to Y$ będzie operatorem liniowym z dziedziną $\mathcal{D}(A)$, czyli dla dowolnych $x,y \in \mathcal{D}(A)$ i dla dowolnych $a, b\in \mathbb{K}$ zachodzi
\begin{displaymath}
	A(ax+by)= aAx + bAy,
\end{displaymath}
gdzie $\mathbb{K}$ jest ciałem nad którym zdefiniowana są przestrzenie $X$ i $Y$.\par 
Jeżeli operator $A$ posiada własność, że $\overline{\mathcal{D}(A)} = X$ to powiemy, że operator $A$ jest gęsto określony w przestrzeni $X$. Z dowolnym operatorem liniowym $A$ o dziedzinie $\mathcal{D}(A)$ można powiązań trzy charakterystyczne zbiory:
\begin{itemize}
	\item $\mathcal{R}(A) = A\mathcal{D}(A)$, czyli obraz operatora $A$,
	\item $\mathcal{N}(A) = \{x\in X\colon Ax= 0\}$, czyli jądro operatora $A$,
	\item $\mathcal{G}(A) = \{(x,y) \in X\times Y\colon x\in\mathcal{D}(A), y = Ax\in \mathcal{R}(A)\}$, czyli wykres operatora $A$.
\end{itemize}
Zauważmy, że w ogólnym przypadku poprawne zdefiniowanie działania w klasie operatorów może być dużym wyzwaniem, jak choćby zdefiniowanie dodawania operatorów $A+B$, gdyż wymaga on prawidłowego zdefiniowania dziedziny nowego operatora - w tym przypadku $\mathcal{D}(A+B) = \mathcal{D}(A)\cap \mathcal{D}(B)$ i może się okazać, że zbiór ten jest pusty. Jeszcze trudniejsze staje się określenie dziadziny, jeżeli chcemy rozważać iloczyn operatorów $AB$, gdyż w takim przypadku musi zachodzić, że $\mathcal{D}(AB) = \{x\in \mathcal{D}(B)\colon Bx \in \mathcal{D}(A)\}$. Ilość i różnorodność operatorów liniowych powoduje, że bardzo trudno jest dowodzić ogólnych własność i twierdzeń dla nich wszystkich.\par 
Wśród operatorów liniowych można wyróżnić specjalna klasę tych operatorów - operatory domknięte. Klasa ta obejmuje bardzo szerokie spektrum różnych operatorów jest jednak wystarczająco wąska by móc wyprowadzać ogólne własność dla wchodzących w jej skład elementów. 

\begin{df}
	Operator liniowy $A\colon X\to Y$ określony na dziedzinie $\mathcal{D}(A) \subset X$, gdzie $X$ i $Y$ są przestrzeniami Banacha, nazwiemy operatorem domkniętym jeżeli zachodzi następujący warunek
	\begin{displaymath}
		\left(\forall_{(x_n) \subset \mathcal{D}(A)}\colon x_n \stackrel{X}{\to}x \land \ Ax_n \stackrel{Y}{\to }y\right) \implies x\in \mathcal{D}(A) \land Ax=y,
	\end{displaymath}
	gdzie zapis $t_n\stackrel{T	}{\to} t$ oznacza, że ciąg $(t_n)$ jest zbieżny do elementu $t$ w normie przestrzeni $T$.
\end{df}
Definicja ta mówi, że operator jest domknięty, jeżeli dla dowolnego zbieżnego ciągu argumentów dla którego ciąg wartości jest zbieżny, musi zachodzić, że granica rozważanego ciągu argumentów, również jest argumentem tego operatora oraz wartość operatora na tym elemencie jest tym samym co granica ciągu wartości na rozważanym ciągu argumentów. Definicja ta wydaje się być bardzo zbliżona do definicji ciągłości operatora, jednak klasa operatorów domkniętych okazuje się być klasą istotnie szerszą niż klasa operatorów ciągłych, co pokazuje przykład poniżej.

\begin{prz}
	Rozważmy przestrzeń funkcji ciągłych na odcinku $[a,b]$ z normą Kołmogorowa $\left(C[a,b], ||\cdot||_{\infty}\right)$ oraz rozważmy operator różniczkowania $A = \frac{d}{dt}\colon C[a,b] \to C[a,b]$ z dziedziną $\mathcal{D}(A) = C^1[a,b]$. Rozważmy ponadto ciąg funkcji $f_n(t) = t^n$. Zauważmy, że ciąg ten zawiera się w dziedzinie operatora $A$ zatem, możemy wyznaczyć ciąg $Af_n$, który przyjmuje postać $Af_n(t) = nt^{n-1}$. Wraz ze wzrostem $n$ norma (równa $n$) tego wyrażenia przyjmuje dowolnie duże wartości, zatem tak zdefiniowany operator $A$ nie jest operatorem ograniczonym, zatem nie może też być operatorem ciągłym. Zauważmy jednak, że z twierdzenia o różniczkowaniu ciągu funkcji operator ten jest domknięty, gdyż zbieżność w zadanej na początku normie Kołomogorowa implikuje zbieżność jednostajną.
\end{prz}

Warunek definiujący operator domknięty może być w wielu przypadkach trudny do sprawdzenia dla dowolnego zbieżnego ciągu argumentów stąd pojawia się potrzeba zastąpienia tego warunku warunkiem równoważnym łatwiejszym do sprawdzenia w niektórych przypadkach. Wykorzystamy do tego celu wprowadzone wcześniej pojęcie wykresu operatora.

\begin{tw}
	Operator $A\colon X\to Y$ jest operatorem domkniętym wtedy i tylko wtedy, gdy jego wykres jest zbiorem domkniętym (w topologii produktowej $X\times Y$).
\end{tw}
\begin{proof}
	Implikacja dowodząca, że wykres operatora domkniętego jest zbiorem domkniętym jest prostą konsekwencją definicji, którą można zapisać w następujący sposób dla dowolnego ciągu argumentów $(x_n)$
	\begin{displaymath}
		\left({x_n} \atop {Ax_n}\right) \to \left(x \atop y\right)\implies \left(x \atop y\right) \in \mathcal{G}(A),
	\end{displaymath}
	czyli $\mathcal{G}(A)$ jest zbiorem domkniętym. By dowieść w przeciwną stronę zauważmy, że fakt, że $\mathcal{G}(A)$ jest zbiorem domkniętym, przy założonym poprzedniku implikacji występującej w definicji operatora domkniętego, implikuje, że $(x, y) \in \mathcal{G}(A)$, a zatem $Ax = y$.
\end{proof}

Poniższe twierdzenie przedstawia własności operatorów domkniętych.

\begin{tw}
	\begin{itemize}
		\item Niech $A$ będzie operatorem domkniętym, $\lambda$ dowolną liczbą zespoloną i niech $I$ oznacza operator identycznościowy. Wtedy operator $A - \lambda I $ też jest operatorem domkniętym.
		\item Jeśli $A$ jest operatorem domkniętym, to jego jądro jest zbiorem domkniętym w $X$.
		\item Jeśli operator $A$ jest domknięty i iniektywny, wtedy operator odwrotny $A^{-1}$ też jest operatorem domkniętym.
	\end{itemize}
\end{tw}

\begin{proof}
	\begin{itemize}
		\item Niech ciąg $(x_n)$ spełnia warunek z definicji operatora domkniętego dla operatora $A$, wtedy mamy 
		\begin{displaymath}
			x_n \stackrel{X	}{\to} x \land Ax_n -\lambda Ix_n \stackrel{X}{\to}y-\lambda  = (A-\lambda I)x.
		\end{displaymath}
		\item niech $(x_n)$ będzie dowolnym ciągiem z jądra operatora $A$, wtedy mamy
		\begin{displaymath}
		0 = Ax_n \stackrel{X}{\to} y
		\end{displaymath}
		co implikuje, że $y = 0$ oraz $Ax = y =0$, zatem jądro operatora $A$ jest domknięte.
		\item Jeżeli operator $A$ jest operatorem iniektywnym, to wtedy 
		\begin{displaymath}
		\mathcal{G}(A) = \{(x,y)\in X\times Y\colon x\in \mathcal{D}(A), y = Ax\in \mathcal{R}(A)\} 
		\end{displaymath}
		\begin{displaymath}
			=\{(y,x)\in Y\times X\colon y\in \mathcal{D}(A^{-1}), A^{-1}y = x\in \mathcal{R}(A^{-1})\} =  \mathcal{G}(A^{-1}),
		\end{displaymath}
		co przy założenie o domkniętości operatora $A$ implikuje domkniętość jego wykresu a zatem również wykresu operatora odwrotnego, co z kolei implikuje jego domkniętość.
	\end{itemize}
\end{proof}

Operatory domknięte można scharakteryzować także przy pomocy odpowiednio zdefiniowanej przestrzeni metrycznej. Intuicyjnie przestrzeń ta powinna być tak skonstruowana, by zdefiniowana w niej metryka mierzyła równocześnie zachowanie argumentów jak i wartości operatora na tych argumentach oraz zapewniała prawidłowe zdefiniowanie działania operatora na badanych elementach. Niech zatem $A\colon X\to Y$ będzie operatorem liniowym z dziedziną $\mathcal{D}(A)\subset X$. Dla dowolnego elementu $x\in \mathcal{D}(A)$ określmy funkcję 
\begin{displaymath}
	||x||_A = ||x||_X + ||Ax||_Y.
\end{displaymath}
Pokażemy, że funkcja ta jest normą na $\mathcal{D}(A)$.
\begin{proof}
	\begin{itemize}
		\item jeżeli $x = 0$, to oczywiście $||x||_A = 0$, ponadto jeżeli $||x||_A = 0$ to musi zachodzić $||x||_X = -||Ax||_Y$ a zatem $||x||_X = 0$, gdyż $||x||_X \geq 0$ oraz $||Ax||_Y \geq 0$, a stąd dostajemy, że $x = 0$,
		\item dla dowolnego skalara $a\in \mathbb{K}$ zachodzi $||ax||_A = ||ax||_X + ||Aax||_y = |a|(||x||_X+ ||Ax||_Y) = |a|||x||_A$,
		\item dla dowolnych dwóch elementów $x,y\in \mathcal{D}(A)$ mamy $||x+y||_A = ||x+y||_X+ ||Ax+Ay||_Y\leq (||x||_X + ||Ax||_Y) + (||y||_X + ||Ay||_Y) = ||x||_A + ||y||_A$.
	\end{itemize}
	Zatem funkcja $||\cdot||_A$ jest normą na $\mathcal{D}(A)$
\end{proof}
Z powyższego faktu oraz tego, że $A$ jest operatorem liniowym dostajemy, że para $(\mathcal{D}(A), ||\cdot||_A)$ jest przestrzenią unormowaną. Będziemy ją oznaczać przez $(X_A, ||\cdot||_A)$.

Poniższe twierdzenie łączy domkniętość operatora z zupełnością przestrzeni $(X_A, ||\cdot||_A)$.
\begin{tw}
	Niech $X,Y$ będą przestrzeniami Banacha oraz niech $A\colon X\to Y$ będzie operatorem liniowym. Wtedy następujące warunki są równoważne
	\begin{itemize}
		\item $A$ jest operatorem domkniętym,
		\item $\mathcal{G}(A)$ jest zbiorem domkniętym w $X\times Y$,
		\item przestrzeń $(X_A, ||\cdot||_A)$ jest przestrzenią zupełną.
	\end{itemize}
\end{tw}

Powyższe twierdzenie sugeruje, że dla pewnej klasy operatorów, które nie są domknięte, można w pewien sposób przywrócić im tą pożądaną własność. Zauważmy, że jeżeli operator $T$ nie jest operatorem domkniętym, to jego wykres nie jest zbiorem domkniętym w $X\times Y$. Zbiór ten  można jednak domknąć i jeżeli okaże się, że tak uzyskany zbiór jest wykresem pewnego operatora, to uzyskaliśmy pewne rozszerzenie operatora $T$, które jest już domknięte. Rozważanie to prowadzi do następującej klasy operatorów.
\begin{df}
	Niech $X,Y$ będą przestrzeniami Banacha i niech $T\colon X\to Y$ z dziedziną $\mathcal{D}(T) \subset X$. Operator $T$ nazwiemy operatorem domykalnym, jeżeli jeżeli istniej pewien operator domknięty $S\colon X\to Y$ będący rozszerzeniem operatora $T$, czyli $\mathcal{D}(T) \subset \mathcal{D}(S)$ oraz $Sx=Tx$ dla dowolnego $x\in \mathcal{D}(T)$.
\end{df}

Najmniejszy spośród operatorów domykających operator $T$ nazywamy domknięciem operatora $T$ i oznaczamy $\bar{T}$.

Poniższe twierdzenie formalizuje przedstawioną wyżej intuicję.

\begin{tw}
	Niech $X,Y$ będą przestrzeniami Banacha i niech $T\colon X\to Y$ z dziedziną $\mathcal{D}(T) \subset X$. Wtedy następujące warunki są równoważne
	\begin{itemize}
		\item $T$ jest operatorem domykalnym,
		\item $\overline{\mathcal{G}(A)} = \mathcal{G}(\bar{T})$,
		\item dla dowolnego ciągu $(x_n) \subset \mathcal{D}(A)$ jeżeli $x_n \stackrel{X}{\to} 0$ oraz $Ax_n\stackrel{Y}{\to} y$, to $y = 0$,
		\item $\widetilde{\left(\mathcal{D}(A), ||\cdot||_A\right)} = \left(\mathcal{D}(\bar{A}, ||\cdot||_A)\right)$.
	\end{itemize}
\end{tw}

Nasuwa się oczywiście pytanie, czy klasa operatorów domykalnych jest tożsama z klasą wszystkich operatorów liniowych. Poniższy przykład pokazuje, że tak nie jest.

\begin{prz}
	Rozważmy przestrzeń $H = L_p[0,1]$ dla pewnego $1\leq p < \infty$. Określmy operator $W$ jako \begin{displaymath}
		Wu(t) = u(0)\pmb{1}(t),
	\end{displaymath}
	gdzie $\pmb{1}(t)$ oznacza funkcje charakterystyczną zbioru $[0,1]$ i niech $\mathcal{D}(W) = C[0,1]$. Niech ciąg funkcji $u_n$ będzie zdefiniowany następująco
	\begin{displaymath}
		u_n(t) = (1-t)^n.
	\end{displaymath}
	Ciąg $(u_n)$ zbiega do $0$ w normie przestrzeni $ L_p[0,1]$, ponieważ 
	\begin{displaymath}
		\int_0^1\left((1-t)^n - 0\right)^pdt = \frac{1}{pn+1} \stackrel{n\to \infty}{\to } 0.
	\end{displaymath}
	Mamy jednak, że 
	\begin{displaymath}
		Wu_n(t) = u(0)\pmb{1}(t) = 1 \neq 0.
	\end{displaymath}
	Zatem operator $W$ nie jest operatorem domykalnym na bazie twierdzenia 4.
\end{prz}

Na koniec wprowadzimy pojęcia rdzenia dla operatora liniowego $A$ oraz wypowiemy związane z nim twierdzenie niosące duże konsekwencje dla operatorów domkniętych.

\begin{df}
	Zbiór $D$ nazwiemy rdzeniem dla operatora $A$, jeżeli jest on zbiorem gęstym w przestrzeni $\left(\mathcal{D}(A), ||\cdot ||_A\right)$.
\end{df}

\begin{tw}
	Niech $A\colon X\to Y$ będzie operatorem domkniętym określonym na przestrzeniach Banacha $X$ i $Y$ z dziedziną $\mathcal{D}(A)$ i niech zbiór $D$ będzie rdzeniem dla tego operatora. Wtedy zachodzi
	\begin{displaymath}
		A = \overline{A|_{D}}.
	\end{displaymath}
\end{tw}
Powyższe twierdzenie mówi, że dla operatorów domkniętych, wystarczająca jest jego znajomość tylko na rdzeniu - operator może być "odzyskany" z rdzenia przez operację domknięcia.

\end{document}