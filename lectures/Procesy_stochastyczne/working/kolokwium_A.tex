\documentclass{mwart}
%\usepackage{polski}
%\usepackage[polish]{babel}
\usepackage{amsfonts}
\usepackage{indentfirst}
\usepackage[utf8]{inputenc}
\usepackage{amsthm}
\usepackage{multirow}
\usepackage{amsmath}
\newtheorem{tw}{Twierdzenie}
\newtheorem{df}{Definicja}
\newtheorem{zd}{Zadanie}
\newtheorem{zdt}[zd]{Zadanie*}
\title{Kolokwium 1\\Grupa A}
\usepackage{Sweave}
\begin{document}
\Sconcordance{concordance:kolokwium_A.tex:kolokwium_A.Rnw:%
1 14 1 1 0 22 1}

\maketitle
\begin{zd}
Niech $T, S$ będą momentami stopu z czasem dyskretnym względem tej samej filtracji. Udowodnij, że momentem stopu jest zmienna losowa $T + S$.
\end{zd}
\begin{zd}
Niech proces $X$ będzie submartyngałem względem pewnej filtracji z czasem dyskretnym o wartości oczekiwanej stałej w czasie. Udowodnij, że proces $X$ jest martyngałem względem tej filtracji.
\end{zd}
\begin{zd}
Niech $X_i$ będą niezależnymi zmiennymi losowymi takimi, że dla każdego $k$ naturalnego dodatniego zachodzi $\mathbb{P}(X_k=k)=\frac{1}{k}$ oraz $\mathbb{P}(X_k = 0)= \frac{k-1}{k}$. Niech $M_n = \prod_{i=1}^nX_i,\ M_0=0$. Udowodnij, że proces $M$ jest martyngałem względem filtracji genereowanej przez zmienne $X_i$.
\end{zd}
\begin{zd}
Niech $A_1, A_2, \dots$ będzie ciągiem zdarzeń niezależnych zdzrzeń i niech $S_n = \sum_{i =1}^n(1-\pmb{1}_{A_i})$. Niech $\{a_n\}$ będzie ciągiem deterministycznym. Jaką postać musi mieć ciąg $\{a_n\}$ by proces $\{S_n + a_n\}$ był martyngałem względem filtracji generowanej przez zmienne $\pmb{1}_{A_i}$?
\end{zd}
\begin{zd}
\begin{itemize}
\item Podaj definicję filtracji generowanej przez proces stochastyczny.
\item Podaj definicję submartyngału.
\end{itemize}
\end{zd}

\end{document}
