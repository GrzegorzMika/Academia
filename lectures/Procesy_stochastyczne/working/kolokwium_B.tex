\documentclass{mwart}
%\usepackage{polski}
%\usepackage[polish]{babel}
\usepackage{amsfonts}
\usepackage{indentfirst}
\usepackage[utf8]{inputenc}
\usepackage{amsthm}
\usepackage{multirow}
\usepackage{amsmath}
\newtheorem{tw}{Twierdzenie}
\newtheorem{df}{Definicja}
\newtheorem{zd}{Zadanie}
\newtheorem{zdt}[zd]{Zadanie*}
\title{Kolokwium 1\\Grupa B}
\usepackage{Sweave}
\begin{document}
\Sconcordance{concordance:kolokwium_B.tex:kolokwium_B.Rnw:%
1 14 1 1 0 27 1}

\maketitle
\begin{zd}
Rozważmy przestrzeń probabilistyczną $\left([0, 1], \mathcal{B}_{[0, 1]}, \lambda \right)$, gdzie $\lambda$ jest miarą Lebesgue'a. Niech $Y_n(\omega) = \omega^2\cdot \pmb{1}_{[0, 1 - 1/n)} + \pmb{1}_{[1-1/n, 1]}$ oraz niech $X(\omega) = 2\omega$.
\begin{itemize}
\item Wyznacz postać filtracji generowanej przez proces $\{Y_n\}$.
\item Wyznacz postać procesu $X_n = \mathbb{E}\left(X|Y_n\right)$.
\item Czy proces $Y_n = X_n^2$ jest martyngałem?
\end{itemize}
\end{zd}
\begin{zd}
Niech proces $\{X_n\}$ będzie procesem symetrycznego błądzenia losowego z czasem dyskretnym i niech $\{\mathcal{F}_n\}$ oznacza filtrację naturalną tego procesu. Znajdź deterministyczny ciąg $a_n\in \mathbb{R}$ taki, że proces zadany jako $Z_n = X_n^3 + a_nX_n$ jest martyngałem względem filtracji $\{\mathcal{F}_n\}$.
\end{zd}
\begin{zd}
Niech $X_i\in \mathcal{L}^2\left(\Omega\right)$, $\mathcal{F}_n = \sigma\left(X_1, X_2, \dots, X_n\right)$. Załóżmy, że $S_n = X_1 + X_2 + \dots + X_n$ jest martyngałem względem filtracji $\{\mathcal{F}_n\}$. Udowodnij, że $\mathbb{E}X_iX_j = 0$ dla $i\neq j$.
\end{zd}
\begin{zd}
Niech $S, T$ będą momentami stopu względem tej samej filtracji. Udowodnij, że zachodzi $\mathcal{F}_{\min\{T,S\}} = \mathcal{F}_{T} \cap \mathcal{F}_{S}$.
\end{zd}
\begin{zd}
\begin{itemize}
\item Podaj definicję warunkowej wartości oczekiwanej zmiennej losowej względem $\sigma$- ciała.
\item Co to jest trajektoria procesu stochastycznego?
\end{itemize}
\end{zd}

\end{document}
