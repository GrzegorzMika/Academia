\documentclass{mwart}
%\usepackage{polski}
%\usepackage[polish]{babel}
\usepackage{amsfonts}
\usepackage{indentfirst}
\usepackage[utf8]{inputenc}
\usepackage{amsthm}
\usepackage{multirow}
\usepackage{amsmath}
\newtheorem{tw}{Twierdzenie}
\newtheorem{df}{Definicja}
\newtheorem{zd}{Zadanie}
\title{Procesy stochastyczne \\ Zadania numeryczne}
\usepackage{Sweave}
\begin{document}
\Sconcordance{concordance:numeryka.tex:numeryka.Rnw:%
1 13 1 1 0 33 1}

\maketitle
\textbf{Zasady:}
\begin{itemize}
\item Celem zadań przede wszystkim jest motywacja do dalszych poszukiwań i badań w zakresie procesów stochastycznych.
\item Ilość możliwych punktów do uzyskania za dane zadanie podana jest obok jego numerka.
\item Na pełne rozwiązanie składa się opis metody, przeliczony przykład oraz replikowany, działający kod wykorzystany do stworzenia przykładu.
\item Implenetacja możliwa jest w jednym z jązyków: C, C++, Python, R, Julia.
\item W przypadku wykorzystywania dodatkowych bibliotek wymagane jest podanie wykorzystanej wersji.
\item Rozwiązania można nadsyłać do końca semestru.
\end{itemize}
\begin{zd}
Opisz i na wybranym przykładzie pokaż metodę symulacji niejednorodnego procesu Poissona.
\end{zd}
\begin{zd}

\end{zd}
\begin{zd}

\end{zd}
\begin{zd}

\end{zd}
\begin{zd}

\end{zd}
\begin{thebibliography}{10}

\end{thebibliography}
\end{document}
