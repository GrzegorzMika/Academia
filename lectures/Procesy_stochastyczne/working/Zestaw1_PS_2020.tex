\documentclass{mwart}
\usepackage{polski}
\usepackage[polish]{babel}
\usepackage{amsfonts}
\usepackage{indentfirst}
\usepackage[utf8]{inputenc}
\usepackage{amsthm}
\usepackage{multirow}
\usepackage{amsmath}
\newtheorem{tw}{Twierdzenie}
\newtheorem{df}{Definicja}
\newtheorem{zd}{Zadanie}
\newtheorem{zdt}[zd]{Zadanie*}
\title{Procesy stochastyczne\\ Zestaw zadań nr 1}
\usepackage{Sweave}
\begin{document}
\Sconcordance{concordance:Zestaw1_PS_2020.tex:Zestaw1_PS_2020.Rnw:%
1 14 1 1 0 2 1 1 6 33 1 1 3 1 2 29 1 1 7 1 5 1 2 6 1}

\maketitle
\begin{zd}
Niech $X_i\colon \left(\Omega, \mathcal{F}\right) \to \left(\Omega_i, \mathcal{F}_i\right), i = 1, 2, \dots, n$ będą niezależnymi zmiennymi losowymi i nich $g_i\colon  \left(\Omega_i, \mathcal{F}_i\right) \to  \left(\Omega'_i, \mathcal{F}'_i\right), i = 1, 2, \dots, n$ będą mierzalne. Co można powiedzieć na temat niezależności $g_i\circ X_i$?
\end{zd}

\begin{zd}
Niech $f, g$ będą mierzalnymi funkcjami borelowskimi. Udowodnij, że są one niezależne wtedy i tylko wtedy, gdy dla dowolonych $a, b, c, d\in \mathbb{R}$ zachodzi:
\begin{displaymath}
\mu\left(f^{-1}[a, b)\cap g^{-1}[c, d)\right) = \mu\left(f^{-1}[a, b)\right)\cap \mu\left(g^{-1}[c, d)\right)
\end{displaymath}
\end{zd}

\begin{zd}
Niech $X$ będzie dyskretną zmienną losową przyjmującą wartości $1, 2, 3, \dots$ z prawdopodobieństwami odpowiednio $p_1, p_2, p_3, \dots$. Niech $Y$ będzie taką zmienną losową, że gdy $X=n$, $Y$ przyjmuje nieujemne wartości zgodnie z rozkładem o gęstości $f_n$. Znajdź prawdopodobieństwo, że $1\leq X+Y\leq 3$.
\end{zd}

\begin{zd}
Niech $X$ będzie zmienną losową przyjmującą tylko skończenie wiele wartości $x_1, x_2, \dots, x_n$ i niech $Y$ będzie taką zmienną losową, że $Y\in \mathcal{L}_1\left(\Omega\right)$ (tzn. $\mathbb{E}Y < +\infty$). Udowodnij, że zachodzi równość
\begin{displaymath}
\mathbb{E}\left(Y|X = x_i\right) = \frac{1}{\mathbb{P}\left(X = x_i\right)}\int_{\{X = x_i\}}Yd\mathbb{P}.
\end{displaymath}
Co można powiedziec na temat $\mathbb{E}\left(Y|B\right)$, gdzie $B$ jest pewnym mierzalnym zbiorem o dodatniej mierze?
\end{zd}

\begin{zd}
Niech $\left\{B_i\right\}_{i =1}^{\infty}$ będzie rodziną wzajemnie rozłącznych zbiorów takich, że $\bigcup_{i =1}^{\infty}B_i=\Omega$ i niech $X\in \mathcal{L}_1\left(\Omega\right)$. Udowodnij, że zachodzi równość
\begin{displaymath}
\mathbb{E}X = \sum_{i =1}^{\infty}\mathbb{P}(B_i)\mathbb{E}\left(X|B_i\right).
\end{displaymath}
\end{zd}

\begin{zd}
Niech $\sigma$-ciało $\mathcal{F}$ będzie generowane przez skończoną rodzinę rozłącznych zbiorów $B_i$ i niech $X\in \mathcal{L}_1\left(\Omega\right)$. Znajdź postać $\mathbb{E}\left(X|\mathcal{F}\right)$.
\end{zd}

\begin{zd}
Niech $X$ będzie nieujemną zmienną losową o skonczonym pierwszym momencie. Pokaż, że zachodzi
\begin{enumerate}
\item $\mathbb{E}\left(X|\mathcal{G}\right)= \int_0^{+\infty}\mathbb{P}\left(X > t\right)dt$,
\item $\mathbb{P}\left(X > \alpha\right)\leq \alpha^{-1}\mathbb{E}\left(X|\mathcal{G}\right)$.
\end{enumerate}
\end{zd}

\begin{zd}
Niech $X \in \mathcal{L}_2\left(\Omega\right)$. Zdefinujmy warunkową wariancję względem $\sigma$- ciała $\mathcal{G}$ w nastepujący sposób
\begin{displaymath}
\mathbb{V}\left(X|\mathcal{G}\right) = \mathbb{E}\left(\left(X - \mathbb{E}\left(X|\mathcal{G}\right)\right)^2|\mathcal{G}\right).
\end{displaymath}
Wykaż, że zachodzi
\begin{displaymath}
\mathbb{V}\left(X\right) = \mathbb{E}\left(\mathbb{V}\left(X|\mathcal{G}\right)\right) + \mathbb{V}\left(\mathbb{E}\left(X|\mathcal{G}\right)\right).
\end{displaymath}
\end{zd}

\begin{zd}
Niech $\Omega = [0,1]\times[0,1]$ i niech $\mathbb{P}$ będzie miarą Lebesgue'a na tej przestrzeni. Wyznacz warunkową wartość oczekiwaną $\mathbb{E}\left(f|\mathcal{G}\right)$, jeśli:
\begin{enumerate}
\item $f(x,y) = x,\ \mathcal{G} = \sigma(y)$,
\item $f(x,y) = x - y,\ \mathcal{G} = \sigma(x+y)$.
\end{enumerate}
\end{zd}
\begin{zd}
Niech $\Omega = [0, 1]$ i niech $\mathbb{P}$ będzie miarą Lebesgue'a na tej przestrzeni. Niech $f(x) = x^2$ i $g(x) = 2\pmb{1}_{[0, 1/2)} + x\pmb{1}_{[1/2, 1]}$. Znajdź $\mathbb{E}\left(f|g\right)$.
\end{zd}
% 2.6 Basic stochastic processes Course through Exercises
\begin{zd}
	Niech $(\Omega = [0,1], \mathcal{F} = \mathcal{B}_{[0,1]}, \lambda)$ będzie przestrzenią probabilistyczną. Niech $Y(\omega) = \omega(1-\omega)$. Udowodnij, że dla dowolnej zmiennej losowej $X$ określonej na tej przestrzeni zachodzi
	\begin{displaymath}
	 \mathbb{E}(X|Y)(\omega) = \frac{X(\omega) + X(1-\omega)}{2}.
	\end{displaymath}
\end{zd}

\begin{zd}
	Niech $X_1, X_2, \dots $ będzie ciągiem niezależnych i całkowalnych zmiennych losowych o tym samym rozkładzie normalnym ($\mathcal{N}(\mu,\sigma)$) i niech $\tau$ będzie zmienną losową o rozkładzie Poissona z parametrem $\lambda$ niezależną od tego ciągu. Znajdź wartość oczekiwaną zmiennej losowej
	\begin{displaymath}
	\xi \stackrel{d}{=} \sum_{n=1}^\tau X_n.
	\end{displaymath}
\end{zd}

\begin{zdt}
Niech $\left(\Omega, \mathcal{F}, \mathbb{P}\right)$ będzie przestrzenią probabilistyczną i niech $\mathcal{G}$ będzie pod-$\sigma$-ciałem ciała $\mathcal{F}$. Niech $Y\in \mathcal{L}_1\left(\Omega\right)$. Wtedy mamy, że $\mathbb{E}\left(Y|\mathcal{G}\right) \in \mathcal{L}_1\left(\Omega\right)$, a zatem $A(Y) = \mathbb{E}\left(Y|\mathcal{G}\right)$ definuje liniowy operator na przestrzeni $\mathcal{L}_1\left(\Omega\right)$. Wykaż, że
\begin{enumerate}
\item $||A|| = 1$,
\item Definując iloczyn skalarny jako $[X, Y] = \int_{\Omega}XYd\mathbb{P}$ wykaż, że $A$ jest samosprzężony.
\end{enumerate}
\end{zdt}



\end{document}
