\documentclass[12pt]{article}
\usepackage{polski}
\usepackage[polish]{babel}
\usepackage{amsfonts}
\usepackage{indentfirst}
\usepackage[utf8]{inputenc}
\usepackage{amsthm}
\usepackage{multirow}
\usepackage{amsmath}
\newtheorem{tw}{Twierdzenie}
\newtheorem{df}{Definicja}
\newtheorem{zd}{Zadanie}
\title{Procesy stochastyczne\\ Zestaw zadań nr 5}
\begin{document}
\maketitle

\begin{zd}
	Mamy rurę długości $30$ centymetrów i z jej lewego końca stawiamy pchłę. Pchła co sekundę wykonuje skok, przy czym pierwszy skok zawsze jest wgłąb rury, jeśli pchła dojdzie do prawego końca rury, to zawsze skacze w lewo, zaś jeśli jest gdzieś w środku, to skacze w lewo lub w prawo z równym prawdopodobieństwem. Skok pchły ma długość $1$cm. Gdy pchła wróci na lewy koniec, łapiemy ją. Ile średnio będzie musieli czekać? W jakich stanach może znaleźć się mucha ( z punktu widzenia teorii łańcuchów Markowa oczywiście).
\end{zd}
\begin{zd}
	Student raz w tygodniu bierze udział w zajęciach z procesów stochastycznych. Na każde zajęcia przychodzi przygotowany lub nie. Jeśli w danym tygodniu jest przygotowany, to w następnym jest przygotowany z prawdopodobieństwem $0.7$. Jeśli natomiast w danym tygodniu nie jest przygotowany, to w następnym jest przygotowany z prawdopodobieństwem $0.2$. Opisz stany w jakich może znaleźć się student i wyznacz macierz przejścia. Na dłuższą metę (bardzo długą), jak często student jest przygotowany?
\end{zd}
\begin{zd}
	Niech $N_t$ będzie procesem Poissona z intensywnością $\lambda$. Udowodnij
	\begin{displaymath}
	\lim_{t\to \infty}\frac{N_t}{t} = \lambda\ p.n.
	\end{displaymath}
\end{zd}
\begin{zd}
	Udowodnij, że suma niezależnych procesów Poissona jest procesem Poissona.
\end{zd}
\begin{zd}
Niech $N_t$ będzie procesem Poissona z intensywnością $\lambda$. Znajdź postać funkcji kowariancji tego procesu
\begin{displaymath}
C_N(t,s) = Cov(N_t, N_s) 
\end{displaymath}
oraz funkcję autokorelacji tego procesu
\begin{displaymath}
	A_N(t,s) = \rho\left(N_t, N_s\right).
\end{displaymath}
\end{zd}
\begin{zd}
	Niech $N_t$ będzie procesem Poissona z intensywnością $\lambda$ i niech $X_1$ będzie czasem pierwszego przybycia. Pokaż, że warunkowo względem zdarzenia $N(t) = 1$, $X_1$ ma rozkład jednostajny na odcinku $(0,t]$, czyli
	\begin{displaymath}
	\mathbb{P}\left(X_1 \leq x|N(t) = 1\right) = \frac{x}{t},\ 0\leq x\leq t.
	\end{displaymath}
\end{zd}
\begin{zd}
	Załóżmy, że $N_t$ jest procesem Poissona, a $Y_1, Y_2, \dots $jest ciągiem nizelażenych zmiennych losowych o jednakowym rozkładzie niezależnym od $N$. Niech 
	\begin{displaymath}
	X_t \stackrel{d}{=} \sum_{k = 1}^{N_t}Y_k,\ N_t > 0.
	\end{displaymath}
	Wykaż, że $\{X_t\}_t $ jest procesem o niezależnych i stacjonarnych przyrostach.
\end{zd}
\end{document}