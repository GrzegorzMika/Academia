\documentclass[12pt]{article}
\usepackage{polski}
\usepackage[polish]{babel}
\usepackage{amsfonts}
\usepackage{indentfirst}
\usepackage[utf8]{inputenc}
\usepackage{amsthm}
\usepackage{multirow}
\usepackage{amsmath}
\newtheorem{tw}{Twierdzenie}
\newtheorem{df}{Definicja}
\newtheorem{zd}{Zadanie}
\title{Procesy stochastyczne\\ Zestaw zadań nr 6}
\begin{document}
\maketitle
Wszędzie dalej $\{W_t\}_{t\in T}$ jest procesem Wienera.
\begin{zd}
	Pokaż, że następujące procesy są martyngałami
	\begin{itemize}
		\item $W_t$, 
		\item $W_t^2 - t$,
		\item $\exp\left(\sigma W_t - \frac{\sigma^2t}{2}\right),\ \sigma > 0$.
	\end{itemize}
\end{zd}
\begin{zd}
	Oblicz
	\begin{itemize}
		\item $\mathbb{P}(W_s < W_t)$,
		\item $\mathbb{P}(0 < W_2 < W_3)$
		\item $\mathbb{E}W_1W_2^2$
		\item $\mathbb{E}\left(W_2^2(W_3 - W_1)\right)$.
	\end{itemize}
\end{zd}

\begin{zd}
	Oblicz $\mathbb{E}\left(W_sW_t^2|\mathcal{F}_{\alpha}\right)$ dla $0\leq \alpha\leq \beta$ i $s,t \in [\alpha, \beta]$.
\end{zd}
\begin{zd}
	Znajdź funkcję kowariancji jednowymiarowego procesu Wienera.
\end{zd}
\begin{zd}
	Określmy następujący proces (most Browna)
	\begin{displaymath}
		B_t = W_t - tW_1, \ t\in[0,1].
	\end{displaymath}
	Sprawdź, czy jest on martyngałem i znajdź jego funkcję kowariancji.
\end{zd}
\begin{zd}
	Udowodnij, że zachodzi 
	\begin{displaymath}
	\lim_{t \to \infty}\frac{W_t}{t} = 0\ p.n.
	\end{displaymath}
\end{zd}
\begin{zd}
	Niech $W_t = (W_t^1, W_t^2)$, gdzie $\{W_t^1\}, \{W_t^2\}$ są niezależnymi procesami Wienera. Dla $R > 0$ i $t > 0$ znajdź $\mathbb{P}\left(||W_t|| < R	\right)$, gdzie $||\cdot||$ jest normą euklidesową w $\mathbb{R}^2$. Oblicz $\lim_{t \to \infty}\mathbb{P}\left(||W_t|| < R	\right)$.
\end{zd}
\begin{zd}
	Sprawdź, czy następujące procesy są procesami Wienera:
	\begin{itemize}
		\item $-W_t$,
		\item $c^{-1/2}W_{ct}$, $c>0$,
		\item $Y_t = tW_{1/t},\ t> 0$ i $Y_0 = 0$,
		\item $W_{T+t} - W_T, \ T > 0$.
	\end{itemize}
\end{zd}
\begin{zd}
	Niech $\{L_n\}$ będzie ciągiem niezależnych zmiennych losowych takich, że $\mathbb{P}\left(L_n = 1\right) = \mathbb{P}\left(L_n = -1\right)$ dla każdego $n\in\mathbb{N}$. Określmy $h = \frac{1}{N}$ oraz proces $X_n = h^{\alpha}L_n$. Pokaż, że dla $\alpha \in (0,1/2)$ $\sum_{k=1}^NX_k \to 0$ w $L_2$, natomiast dla $\alpha \in (1/2, \infty)$ $\sum_{k=1}^NX_k \to \infty$ w $L_2$. Co dostajemy dla $\alpha = 1/2$?
\end{zd}
\begin{zd}
	Pokaż, że proces Wienera jako funkcja $W\colon [0,\infty) \to \L_2(\Omega)$ jest funkcją ciągłą i nigdzie nieróżniczkowalną.
\end{zd}
\begin{zd}
	Pokaż, że proces określony jako $Z_t = \sqrt{t}N(0,1)$ nie jest procesem Wienera.
\end{zd}
\begin{zd}
	Udowodnij, że dla prawie wszystkich trajektorii procesu Wienera, zachodzi
	\begin{displaymath}
		\sup_tW_t = +\infty\ \inf_tW_t = -\infty.
	\end{displaymath}
\end{zd}


\end{document}