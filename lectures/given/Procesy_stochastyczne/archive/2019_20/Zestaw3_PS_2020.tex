\documentclass{mwart}
%\usepackage{polski}
%\usepackage[polish]{babel}
\usepackage{amsfonts}
\usepackage{indentfirst}
\usepackage[utf8]{inputenc}
\usepackage{amsthm}
\usepackage{multirow}
\usepackage{amsmath}
\newtheorem{tw}{Twierdzenie}
\newtheorem{df}{Definicja}
\newtheorem{zd}{Zadanie}
\newtheorem{zdt}[zd]{Zadanie*}
\title{Procesy stochastyczne\\ Zestaw zadań nr 3}
\usepackage{Sweave}
\begin{document}
\Sconcordance{concordance:Zestaw3_PS_2020.tex:Zestaw3_PS_2020.Rnw:%
1 14 1 1 0 66 1}

\maketitle

\begin{zd}
Znajdź postać filtracji generowanej przez proces $X(n, \omega) = \omega^2\pmb{1}_{[0, 2+1/n]}$.
\end{zd}

\begin{zd}
Niech dana będzie filtracja $\{\mathcal{F}_n\}$ i całkowalna zmienna losowa $X$. Udowodnij, że martyngałem względem tej filtracji jest proces określony
następująco
\begin{displaymath}
\mathbb{E}\left(X|\mathcal{F}_m\right),\ m>0.
\end{displaymath}
\end{zd}

\begin{zd}
	Określmy proces $Z(n)$ w następujący sposób
	\begin{displaymath}
		Z(n) = Z(n-1) + L(n), Z(0) = 0,\  \mathbb{P}(L(n) = 1)  = \mathbb{P}(L(n) = -1),
	\end{displaymath}
	gdzie zmienne $L(n)$ są niezależne między sobą.\\
	Udowodnij, że następujące procesy są martyngałami względem filtracji $\mathcal{F}_n = \sigma\left(Z(0), Z(1), \dots, Z(n)\right)$:
	\begin{itemize}
		\item $Z(n), \ n= 0, 1, \dots$
		\item $Z(n)^2 - n, \ n= 0, 1, \dots$,
		\item $ (-1)^n\cos \left(\pi Z(n)\right), \ n= 0, 1, \dots.$
	\end{itemize}
\end{zd}

\begin{zd}
Niech $\{X_n\}_{n=1}^{\infty}$ będzie ciągiem niezależnych zmiennych losowych o tym samym rozkładzie $\mathcal{N}\left(-a, 1\right),\ a >0$.
\begin{itemize}
\item Dla jakiej wartości $h\in \mathbb{R}$ proces $Y_n = \exp \left(h\sum_{i = 1}^nX_i\right)$ jest martyngałem względem swojej filtracji naturalnej?
\item Dla jakich wartości $h$ proces ten jest sub- lub supermartyngałem?
\item Niech $h=2a$ i niech $x > 0$. Określmy $S_n = \sum_{i=1}^nX_i$. Udowodnij, że zachodzi
\begin{displaymath}
\mathbb{P}\left(\sup_nS_n > x\right) \leq e^{-2ax}.
\end{displaymath}
\end{itemize}
\end{zd}


\begin{zd}
Niech $X_1, X_2, X_3, \dots$ będą niezależnymi, całkowalnymi zmiennymi losowymi o wartości oczekiwanej równej 1. Niech $S_n = X_1\cdot X_2\cdot \dots \cdot X_n$. Udowodnij, że $S_n$ jest martyngałem względem filtracji generowanej przez zmienne $X_i$ .
\end{zd}

\begin{zd}
Niech $X_1, X_2, X_3, \dots$ będzie ciągiem niezależnych zmiennych losowych o tym samym rozkładzie, odpowiednio całkowalnych i o średniej zero. Niech $S_n = X_1+X_2+\dots + X_n$. Pokaż, że proces $S_n^2 - n\mathbb{E}X_1^2$  jest martyngałem. Czy proces $S_n^3$ jest martyngałem? Jaką postać komensacji $A_n$ należy zaproponować, by proces $S_n^3-A_n$ był martyngałem? Co gdy $S_n^3$ zastąpimy przez $S_n^m$?
\end{zd}

\begin{zd}
Niech $X_1, X_2, X_3, \dots$ będzie ciągiem niezależnych zmiennych losowych o tym samym rozkładzie zadanym przez
\begin{displaymath}
\mathbb{P}\left(X_i = 1\right) = p = 1-\mathbb{P}\left(X_i = -1\right) = 1-q
\end{displaymath}
oraz niech $\{\mathcal{F}_n\}$ będzie filtracją generowaną przez zmienne losowe $X_i$. Niech $S_n = X_1 + X_2 + \dots +X_n$. Udowowdnij
\begin{itemize}
\item $M_n = \left(q/p\right)^{S_n}$ jest martyngałem względem $\{\mathcal{F}_n\}$,
\item dla $\lambda >0 $ wyznacz stałą $C = C(\lambda)$ taką, że proces $Z_n^{\lambda} = C^n\lambda^{S_n}$ jest martyngałem względem $\{\mathcal{F}_n\}$.
\end{itemize}
\end{zd}


\begin{zd}
Udowodnij, że suma procesów będących martyngałem względem tej samej filtracji jest martyngałem względem tej filtracji. Co gdy procesy te są martyngałami względem różnych filtracji?
\end{zd}

\begin{zd}
Udowodnij, że wartość oczekiwana martyngału względem zadanej filtracji jest stała w czasie.
\end{zd}

\begin{df}
Proces $X$ nazywamy przewidywalnym względem filtracji $\{\mathcal{F}_t\}_{n\in \mathbb{N}}$, jeżeli dla dowolonego $n$ $X_{n+1}$ jest $\mathcal{F}_{n}$ mierzalna.
\end{df}

\begin{zd}
Niech $M_t$ będzie martyngałem całkowalnym z kwadratem względem filtracji $\{\mathcal{F}_n\}_{n=1}^{\infty}$. Niech proces $Y_t$ będzie skonstruowany na tej samej przestrzeni probabilistycznej co proces $M_t$ i niech będzie procesem przewidywalnym. Określmy proces $N$ jako
\begin{displaymath}
N_t = N_0 + \sum_{k=1}^tY_k\left(M_k - M_{k-1}\right).
\end{displaymath}
Udowodnij, że proces $N$ jest martyngałem pod warunkiem, że $N_0$ jest $\mathcal{F}_0$-mierzalna.\\
Co trzeba założyć na temat całkowalności procesu $Y$?
\end{zd}

\begin{df}
Proces $X$ nazyway submartyngałem (supermartyngałem) względem filtracji $\{\mathcal{F}_t\}_{t\in T}$, gdy $\forall_{s <t}\ X_s \leq \ (\geq)\ \mathbb{E}\left(X_t|\mathcal{F}_s\right)$.
\end{df}

\begin{zd}
Wykazać, że przy odpowiednich założeniach co do całkowalności funkcja wypukła martyngały względem pewnej filtracji jest submartyngałem względem tej filtracji oraz że funkcja wypukła i niemalejąca przekształca submartyngał w submartyngał.
\end{zd}

\begin{zd}
Udowodnij, że przyrosty martyngału są parami nieskorelowane.
\end{zd}

\begin{zdt}
Pokaż, że filtracja naturalna jest najmniejszą filtracją taką, że dany proces jest do niej adaptowany.
\end{zdt}

\end{document}
