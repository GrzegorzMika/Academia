\documentclass{mwart}
%\usepackage{polski}
%\usepackage[polish]{babel}
\usepackage{amsfonts}
\usepackage{indentfirst}
\usepackage[utf8]{inputenc}
\usepackage{amsthm}
\usepackage{multirow}
\usepackage{amsmath}
\newtheorem{tw}{Twierdzenie}
\newtheorem{df}{Definicja}
\newtheorem{zd}{Zadanie}
\title{Procesy stochastyczne \\ Zadania numeryczne}
\usepackage{Sweave}
\begin{document}
\Sconcordance{concordance:numeryka.tex:numeryka.Rnw:%
1 13 1 1 0 33 1}

\maketitle
\textbf{Zasady:}
\begin{itemize}
\item Celem zadań przede wszystkim jest motywacja do dalszych poszukiwań i badań w zakresie procesów stochastycznych.
\item Ilość możliwych punktów do uzyskania za dane zadanie podana jest obok jego numerka.
\item Na pełne rozwiązanie składa się opis metody, przeliczony przykład oraz replikowany, działający kod wykorzystany do stworzenia przykładu.
\item Implenetacja możliwa jest w jednym z jązyków: C, C++, Python, R, Julia.
\item W przypadku wykorzystywania dodatkowych bibliotek wymagane jest podanie wykorzystanej wersji.
\item Rozwiązania można nadsyłać do końca semestru.
\end{itemize}
\begin{zd}[15 pkt]
Opisz i na wybranym przykładzie pokaż metodę symulacji niejednorodnego procesu Poissona (\it{non-homogeneous Poisson process}) .
\end{zd}
\begin{zd}[10 pkt]
Opisz i na wybranym przykładzie pokaż metodę symulacji dwuwymiarowego jednorodnego procesu Poissona (\it{two-dimensional homogeneous Poisson process}).
\end{zd}
\begin{zd}[15 pkt]
Opisz i na przykładzie przedstaw metodę symulacji jednowymiarowego procesu dyfuzji ze skokami (\it{diffusion-jump process}).
\end{zd}
\begin{zd}[12 pkt]
Opisz i na przykładzie (najlepiej procesu Ornsteina- Uhlenbecka) przdstaw metodę symulacji trajektorii (\it{path}) procesu będącego rozwiązaniem stochastycznego równania różniczkowego (\it{stochastic differential equations, SDE}).
\end{zd}
\begin{zd}[7 pkt]
Zaprezentuj i porównaj metody numerycznego (iteracyjne i nie) wyznaczania rozkładu stacjonarnego dla łańcucha Markowa ze skończoną przestrzenią stanów (\it{stationary distribution of Markov chain}). Jesli to możliwe za[rezentuj graficznie zbieżność rozkładu do rozkładu stacjonarnego.
\end{zd}
\begin{thebibliography}{10}
\bibitem{poisson} Sheldon M. Ross, \emph{Introduction to Probability Models}, Academic Press, 2019.
\bibitem{poisson2} Raghu Pasupathy, \emph{Generating Nonhomogeneous Poisson Processes}, Department of Industrial \& Systems Engineering, Virginia Tech.
\bibitem{sde} Martin Haugh, \emph{Simulating Stochastic Differential Equations}, Monte-Carlo Simulation, Columbia University.
\bibitem{MC} C.C. Paige, George P.H. Styan, Peter G. Wachter, \emph{Computation of the stationary distribution of a markov chain}, Journal of Statistical Computation and Simulation, 1975.
\end{thebibliography}
\end{document}
