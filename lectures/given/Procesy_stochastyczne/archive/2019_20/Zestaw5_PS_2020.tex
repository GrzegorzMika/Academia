\documentclass{mwart}
%\usepackage{polski}
%\usepackage[polish]{babel}
\usepackage{amsfonts}
\usepackage{indentfirst}
\usepackage[utf8]{inputenc}
\usepackage{amsthm}
\usepackage{multirow}
\usepackage{amsmath}
\newtheorem{tw}{Twierdzenie}
\newtheorem{df}{Definicja}
\newtheorem{zd}{Zadanie}
\newtheorem{zdt}[zd]{Zadanie*}
\title{Procesy stochastyczne\\ Zestaw zadań nr 5}
\usepackage{Sweave}
\begin{document}
\Sconcordance{concordance:Zestaw5_PS_2020.tex:Zestaw5_PS_2020.Rnw:%
1 14 1 1 0 61 1}

\maketitle
\begin{zd}
	Niech $X_1, X_2, \dots $ będzie ciągiem niezależnych i całkowalnych zmiennych losowych o tym samym rozkładzie normalnym ($\mathcal{N}(\mu,\sigma)$) i niech $\tau$ będzie zmienną losową o rozkładzie Poissona z parametrem $\lambda$ niezależną od tego ciągu. Znajdź wartość oczekiwaną zmiennej losowej
	\begin{displaymath}
	\xi \stackrel{d}{=} \sum_{n=1}^\tau X_n.
	\end{displaymath}
\end{zd}
\begin{zd}
Niech $X_1, X_2, \dots$ będą niezależnymi zmiennymi losowymi o tym samym rozkładzie, nieujemnymi z wartością oczekiwaną równą $1$. Niech $T$ będzie ograniczonym momentem stopu. Udowodnij, że
\begin{displaymath}
\mathbb{E}\prod_{i=1}^TX_i = 1.
\end{displaymath}
\end{zd}
\begin{zd}
Niech $X_1, X_2, \dots$ będą niezależnymi zmiennymi losowymi o tym samym rozkładzie i niech $\phi$ oznacza funkcję generującą momenty dla $X_i$. Niech ponadto $T$ będzie ograniczonym momentem stopu. Oznaczmy przez $S_T=\sum_{i=1}^TX_i$. Udowodnij, że
\begin{displaymath}
\mathbb{E}\left(\frac{\exp{(\theta S_T)}}{\phi(\theta)^T}\right)=1.
\end{displaymath}
\end{zd}
\begin{zd}
Niech $X_1, X_2, \dots$ będą niezależnymi zmiennymi losowymi o dystrybuancie $F$. Oznaczmy $\tau = \inf\{n: X_n > X_0\}$. Wyznacz rozkład $\tau$ oraz jego wartość oczekiwaną.
\end{zd}
\begin{zd}
Niech $X$ będzie symetrycznym błądzeniem losowym z czasem dyskretnym postaci $X_n = \sum_{i=1}^nY_i$ i niech filtracja $\{\mathcal{F}_n\}$ będzie genereowana przez zmienne $Y_i$. Weźmy dowolne $K\in \mathbb{N}$ i określmy $T = \inf\{n\colon |X_n|=K\}$. Udowdnij:
\begin{itemize}
\item $T$ jest momentem stopu,
\item proces $Z_n = (-1)^n\cos\left(\pi\cdot (X_n+K)\right)$ jest martyngałem,
\item proces $Z$ spełnia założenia twierdzenia o opcjonalnym stopowaniu,
\item znajdź $\mathbb{E}(-1)^T$.
\end{itemize}
\end{zd}
\begin{zd}
Niech $M$ będzie nieujemnym martyngałem takim, że dla dowolnego $n$ $M_n\in L^p$ dla pewnego $p > 1$ i niecch $\lambda > 0$. Udwodnij, że zachodzi
\begin{displaymath}
\mathbb{P}\left(\max_{k\leq n}M_k \geq \lambda \right) \leq \frac{1}{\lambda^p}\int_{\max_{k\leq n}M_k \geq \lambda}M_n^pd\mathbb{P} \leq  \frac{1}{\lambda^p}\mathbb{E}M_n^p.
\end{displaymath}
\end{zd}
\begin{zd}Nierówność Doob'a w $L^p$.
\begin{itemize}
\item Niech $X, Y$ będą nieujemnymi zmiennymi losowymi i niech $Y\in L^p$ dla $p> 1$. Ponadto niech dla dowolnego $x>0$ zachodzi
\begin{displaymath}
x\mathbb{P}\left(X \geq x\right) \leq \int_{X \geq x}Yd\mathbb{P}.
\end{displaymath}
Udowodnij, że $X\in L^p$ oraz $||X||_p \leq \frac{p}{p-1}||Y||_p$.
\item Korzystając z faktu wykazanego powyżej udowodnij, że dla dowolnego nieujemnego submartyngału $M$ takiego, że dla dowolnego $n$ $\mathbb{E}M_n < \infty$ zachodzi
\begin{displaymath}
\left(\mathbb{E}\left(\max_{k\leq n}M_k\right)^p\right)^{1/p} \leq \frac{p}{p-1}\left(\mathbb{E}M_n^p\right)^{1/p}
\end{displaymath}
\end{itemize}
\end{zd}
\begin{zd}
Niech ciągi $\{X_n\}, \{Y_n\}$ oraz zmienna losowa $Y$ bądą określone na tej samej przestrzeni probabilistycznej. Niech ponadto zachodzi $|X_n-Y_n|\stackrel{P}{\to}0$ oraz $Y_n\stackrel{d}{\to}Y$. Udowodnij, że $X_n\stackrel{d}Y$.
\end{zd}
\begin{zd}
Rozważmy ciąg dystrybuant $\{F_n\}$ oraz pewną dystrybunatę $F$. Udowdonij, że ciąg $\{F_n\}$  zbiega słabo do $F$ wtedy i tylko wtedy, gdy dla dowolnego $\epsilon > 0, h> 0$ oraz $t\in suppF$ istniej $N = N(t, h, \epsilon)$ takie, że dla dowlonego $n > N$ zachodzi
\begin{displaymath}
F(t-h) - \epsilon \leq F_n(t)\leq F(t+h)+\epsilon.
\end{displaymath}
\end{zd}
\end{document}
