\documentclass{mwart}
%\usepackage{polski}
%\usepackage[polish]{babel}
\usepackage{amsfonts}
%\usepackage{indentfirst}
\usepackage[utf8]{inputenc}
\usepackage{amsthm}
\usepackage{multirow}
\usepackage{amsmath}
\newtheorem{tw}{Twierdzenie}
\newtheorem{df}{Definicja}
\newtheorem{zd}{Zadanie}
\newtheorem{zdt}[zd]{Zadanie*}
\title{Zestaw 3}
\usepackage{Sweave}
\begin{document}
\Sconcordance{concordance:Zestaw3.tex:Zestaw3.Rnw:%
1 14 1 1 0 97 1}

\maketitle


\begin{zd}
Niech $X_1, X_2, \dots$ będzie ciągiem niezależnych scentrowanych zmiennych losowych o tym samym rozkładzie i niech $\mathcal{F}_n = \sigma(X_1, X_2, \dots, X_n)$. Oznaczmy $S_n = \sum_{i=1}^nX_n$. Pokaż, że $(S_n, \mathcal{F}_n)$ jest martyngałem.
\end{zd}


\begin{zd}
Niech $X$ będzie całkowalną zmienną losową i niech $(\mathcal{F}_n)$ będzie dowolną filtracją. Pokaż, że proces zdefiniowany jako $X_n = \mathbb{E}(X|\mathcal{F}_n)$ jest martyngałem względem filtracji $(\mathcal{F}_n)$.
\end{zd}

\begin{zd}
Niech $Z_n = X_1\cdot X_2 \cdot \dots \cdot X_n$, gdzie $X_i$ są mierzalne względem niezależnych $\sigma$- ciał $\mathcal{G}_n$, $\mathbb{E}X_i = 1$. Oznaczmy $\mathcal{F}_n = \sigma(\mathcal{G}_1, \dots, \mathcal{G}_n)$. Pokaż, że $(Z_n, \mathcal{F}_n)$ jest martyngałem.
\end{zd}

\begin{zd}
Niech $(X_n, \mathcal{F}_n)$ będzie martyngałem, zaś ciąg $(V_n)$ będzie procesem prognozowalnym, czyli takim, ze zmienna losowa $V_n$ jest $\mathcal{F}_{n-1}$ mierzalna, (przyjmujemy $\mathcal{F}_{-1} = \{\emptyset, \Omega\}$). Zakladamy ponadto, ze zmienne losowe $V_n$, sa ograniczone i definiujemy
\begin{displaymath}
Z_n = V_0X_0 + V_1(X_1-X_0) + V_2(X_2-X_1) + \dots + V_n(X_n - X_{n-1}).
\end{displaymath}
Pokaż, że $(Z_n, \mathcal{F}_n)$ jest martyngałem.
\end{zd}

\begin{zd}
Wykazać, że przy odpowiednich założeniach co do całkowalności funkcja wypukła martyngały względem pewnej filtracji jest submartyngałem względem tej filtracji oraz że funkcja wypukła i niemalejąca przekształca submartyngał w submartyngał.
\end{zd}

\begin{zd}
Udowodnij, że przyrosty martyngału są parami nieskorelowane.
\end{zd}

\begin{zd}
	Określmy proces $Z(n)$ w następujący sposób
	\begin{displaymath}
		Z(n) = Z(n-1) + L(n), Z(0) = 0,\  \mathbb{P}(L(n) = 1)  = \mathbb{P}(L(n) = -1),
	\end{displaymath}
	gdzie zmienne $L(n)$ są niezależne między sobą.\\
	Udowodnij, że następujące procesy są martyngałami względem filtracji $\mathcal{F}_n = \sigma\left(Z(0), Z(1), \dots, Z(n)\right)$:
	\begin{itemize}
		\item $Z(n), \ n= 0, 1, \dots$
		\item $Z(n)^2 - n, \ n= 0, 1, \dots$,
		\item $ (-1)^n\cos \left(\pi Z(n)\right), \ n= 0, 1, \dots.$
	\end{itemize}
\end{zd}

\begin{zd}
Niech $\{X_n\}_{n=1}^{\infty}$ będzie ciągiem niezależnych zmiennych losowych o tym samym rozkładzie $\mathcal{N}\left(-a, 1\right),\ a >0$.
\begin{itemize}
\item Dla jakiej wartości $h\in \mathbb{R}$ proces $Y_n = \exp \left(h\sum_{i = 1}^nX_i\right)$ jest martyngałem względem swojej filtracji naturalnej?
\item Dla jakich wartości $h$ proces ten jest sub- lub supermartyngałem?
\item Niech $h=2a$ i niech $x > 0$. Określmy $S_n = \sum_{i=1}^nX_i$. Udowodnij, że zachodzi
\begin{displaymath}
\mathbb{P}\left(\sup_nS_n > x\right) \leq e^{-2ax}.
\end{displaymath}
\end{itemize}
\end{zd}

\begin{zd}
Niech $X_1, X_2, X_3, \dots$ będzie ciągiem niezależnych zmiennych losowych o tym samym rozkładzie, odpowiednio całkowalnych i o średniej zero. Niech $S_n = X_1+X_2+\dots + X_n$. Pokaż, że proces $S_n^2 - n\mathbb{E}X_1^2$  jest martyngałem. Czy proces $S_n^3$ jest martyngałem? Jaką postać komensacji $A_n$ należy zaproponować, by proces $S_n^3-A_n$ był martyngałem? Co gdy $S_n^3$ zastąpimy przez $S_n^m$?
\end{zd}

\begin{zd}
Niech $X_1, X_2, X_3, \dots$ będzie ciągiem niezależnych zmiennych losowych o tym samym rozkładzie zadanym przez
\begin{displaymath}
\mathbb{P}\left(X_i = 1\right) = p = 1-\mathbb{P}\left(X_i = -1\right) = 1-q
\end{displaymath}
oraz niech $\{\mathcal{F}_n\}$ będzie filtracją generowaną przez zmienne losowe $X_i$. Niech $S_n = X_1 + X_2 + \dots +X_n$. Udowowdnij
\begin{itemize}
\item $M_n = \left(q/p\right)^{S_n}$ jest martyngałem względem $\{\mathcal{F}_n\}$,
\item dla $\lambda >0 $ wyznacz stałą $C = C(\lambda)$ taką, że proces $Z_n^{\lambda} = C^n\lambda^{S_n}$ jest martyngałem względem $\{\mathcal{F}_n\}$.
\end{itemize}
\end{zd}


\begin{zd}
Udowodnij, że suma procesów będących martyngałem względem tej samej filtracji jest martyngałem względem tej filtracji. Co gdy procesy te są martyngałami względem różnych filtracji?
\end{zd}

\begin{zd}
Udowodnij, że wartość oczekiwana martyngału względem zadanej filtracji jest stała w czasie.
\end{zd}

\begin{zd}
	Niech $\Omega = [0,1]$. Znajdź postać filtracji generowanej przez proces $X(n,\omega) = 2\omega \chi_{[0, 1 - 1/n]}(\omega)$.
\end{zd}

\begin{zd}
	Pokaż, że filtracja naturalna jest najmniejszą filtracją taką, że dany proces jest do niej adaptowany.
\end{zd}

\begin{zd}
	Niech $\{\xi_n\}$ będzie martyngałem względem pewnej filtracji $\{\mathcal{F}_n\}$. Udowodnij, że $\{\xi_n\}$ jest również martyngałem względem swojej filtracji naturalnej.
\end{zd}

\end{document}
