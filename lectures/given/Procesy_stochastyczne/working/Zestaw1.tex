\documentclass{mwart}
%\usepackage{polski}
%\usepackage[polish]{babel}
\usepackage{amsfonts}
% \usepackage{indentfirst}
\usepackage[utf8]{inputenc}
\usepackage{amsthm}
\usepackage{multirow}
\usepackage{amsmath}
\newtheorem{tw}{Twierdzenie}
\newtheorem{df}{Definicja}
\newtheorem{zd}{Zadanie}
\newtheorem{zdt}[zd]{Zadanie*}
\title{Zestaw 1}
\usepackage{Sweave}
\begin{document}
\Sconcordance{concordance:Zestaw1.tex:Zestaw1.Rnw:%
1 14 1 1 0 46 1}

\maketitle
\begin{zd}
Pokaż, że zmienne losowe $X$, $Y$ są niezależne wtedy i tylko wtedy, gdy niezależne są sigma ciała $\sigma(X)$ i $\sigma(Y)$.
\end{zd}

\begin{zd}
Niech $\mathbb{P}(B)\neq 0$. Pokaż, że zdarzenia $A$ i $B$ są niezależne wtedy i tylko wtedy, gdy $\mathbb{P}(A|B) = \mathbb{P}(A)$.
\end{zd}

\begin{zd}
Wykazać, że jeżeli $\left(X_a)_{a\in A}\right)$ jest ciągiem zmiennych losowych, $A$ jest zbiorem nieprzeliczalnym, to $\sup_AX_a$ nie musi być zmienną losową.
\end{zd}

\begin{zd}
Niech $\mu$ będzie rozkładem prawdopodobieństwa o dystrybuancie $F$, gdzie $F$ jest dana wzorem
\begin{displaymath}
F(x) = (0.1+x)\pmb{1}_{x\in[0; 0.5)} + (0.4+x)\pmb{1}_{x\in[0.5; 0.55)} + \pmb{1}_{x\in[0.55; \infty]}.
\end{displaymath}
Znajdź $\mu(\{0.5\})$, $\mu([0; 0.5])$, $\mu((0; 0.55))$.
\end{zd}

\begin{zd}
Niech $\mu$ będzie rozkładem prawdopodobieństwa na $\mathbb{R}$. Pokaż, że $x_0\in \mathbb{R}$ jest punktem skoku rozkładu $\mu$ wtedy i tylko wtedy, gdy dystrybuantu rozkładu $\mu$ jest nieciągła w $x_0$.
\end{zd}

\begin{zd}
Pokazać, że dowolny rozkład prawdopodobieństwa $\mu$ może mieć co najwyżej przeliczalną liczbę punktów skoku.
\end{zd}

\begin{zd}
Niech $X$ będzie zmienną losową o rozkładzie wykładniczym z parametrem $\alpha$. Wyznacz rozkład zmiennej losowej $Y$ zdanej jako $Y=3X-5$.
\end{zd}

\begin{zd}
Niech zmienna losowa $U$ ma rozkład jednostajny na odcinku $[0, 2]$. Wyznacz rozkład zmiennych $Y=\min(X, X^2)$ i $Z=\max(1, X)$.
\end{zd}

\begin{zd}
Niech zmienna losowa $X$ ma standardowy rozkład normalny. Wyznacz dystrybuantę i gęstośc zmiennych losowych $Y=\exp{X}$ i $Z=X^2$.
\end{zd}

\begin{zd}
Niech $X$będzie nieujemną zminną losową. Udowodnij, że $\mathbb{E}X = \int_{0}^{\infty}\mathbb{P}\left(X \geq t\right)dt$
\end{zd}

\begin{zd}
Niech $X$ będzie zmienną losową o nośniku na liczbach dodatnich całkowitych. Wykaż, że $\mathbb{E}X = \sum_{n=1}^{\infty}\mathbb{P}\left(X \geq n\right)$.
\end{zd}

\begin{zd}
Oblicz wartość oczekiwaną zmiennej losowej o rozkładzie geometrycznym z parametrem $p$.
\end{zd}
\end{document}
