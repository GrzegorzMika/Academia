\documentclass{mwart}
%\usepackage{polski}
%\usepackage[polish]{babel}
\usepackage{amsfonts}
\usepackage{indentfirst}
\usepackage[utf8]{inputenc}
\usepackage{amsthm}
\usepackage{multirow}
\usepackage{amsmath}
\newtheorem{tw}{Twierdzenie}
\newtheorem{df}{Definicja}
\newtheorem{zd}{Zadanie}
\newtheorem{zdt}[zd]{Zadanie*}
\title{Zestaw 2}
\usepackage{Sweave}
\begin{document}
\Sconcordance{concordance:Zestaw2.tex:Zestaw2.Rnw:%
1 14 1 1 0 68 1}

\maketitle

\begin{zd}
Zmienne losowe $X$ i $y$ są niezależne i mają geśtosći odpowiednio $f_1$, $f_2$. Udowodnij, że gęstość zmiennej losowej $Z=X/Y$ wyraża się wzorem $g(u) = \int_{-\infty}^{\infty}|y|f_1(yu)f_2(y)dy$.
\end{zd}

\begin{zd}
Podać przykłady:
\begin{itemize}
\item zmiennych losowych $X_1$, $X_2$ zależnych i skorelowanych,
\item zmiennej losowej $X$ i takich funkcji $f$, $g$, że zmienne $f(X)$ i $g(X)$ są niezależne.
\end{itemize}
\end{zd}

\begin{zd}
Niech $Z=\sum_{n=1}^{\infty}2^{-n}U_n$, gdzie $(U_n)$ jest ciągiem Bernoulliego. Wykaż, że $U\sim [0, 1]$.
\end{zd}

\begin{zd}
Niech $X$, $Y$ będą niezależnymi wektorami losowymi, Wykaż, że $\mathbb{P}\left(X\in A,\ (X, Y)\in B\right) = \int_A\mathbb{P}\left((u, Y)\in B \right)d\mu_X$.
\end{zd}

\begin{zd}
Niech $\mathbb{P}(B) > 0$. Udowodnij, że $\mathbb{P}(A|B)$ jako funkcja $A$, przy ustalonym $B$ jest prawdopodobieństwem
\end{zd}

\begin{zd}
Jest $n$ monet, ale $k$ z nich jest asymetrycznych i orzeł wypada na nich z prawdopodobieństwem $1/3$. Wybrano losowo monetę i w wyniku rzutu wypadł orzeł. Jaka jest szansa, że moneta jest asymetryczna?
\end{zd}

\begin{zd}
W zbiorze 100 monet, jedna ma po obu stronach orły, natomiast pozostałe są prawidłowe. W wyniku 10 rzutów losowo wybraną monetą, otrzymaliśmy 10 orłów. Oblicz prawdopodobieństwo, że była to moneta z dwoma orłami.
\end{zd}

\begin{tw} [Nierówność Bernsteina]
Niech $S_n$ oznacza liczbę sukcesów w schemacie Bernoulliego z prawdopodobieństwem $p$. Wtedy dla dowolnego $\epsilon > 0$, $\mathbb{P}\left(\left|\frac{S_n}{n} -p \right| > \epsilon\right) \leq 2\exp(-n\epsilon^2/4)$.
\end{tw}

\begin{zdt}
Udowodnij nierówność Bernsteina.
\end{zdt}

\begin{zd}
Korzystając z nierówności Bernsteina, udowodnij, że w schemacie Bernoulliego z prawdopodobieństwem $p$, $\frac{S_n}{n}\to p$ prawie na pewno.
\end{zd}

\begin{zd}
Jeśli $VarX_n \leq C < \infty$ dal dowolnego $n$ oraz współczynniki korelacji $\rho (X_i, X_j) \to 0$, gdy $|i-j|\to \infty$, to ciąg $(X_n)$ spełnia SPWL.
\end{zd}

\begin{zd}
Wykaż, że jeśli $(X_n)$ jest ciągiem niezależnych zmiennych losowych o jednakowym rozkładzie takich, że $\mathbb{E}X_1^- < \infty$, $\mathbb{E}X_1^+ = \infty$, to $\mathbb{P}\left(\lim_n\frac{S_n}{n} = \infty\right) = 1$.
\end{zd}

\begin{zd}
Niech $X_1, X_2, \dots \in L^2\left(\Omega \right)$ będą nieskorelowanymi zmiennymi losowymi o wspólnie ograniczonej wariancji. Udowodnij, że wtedy $\frac{X_1+X_2 + \dots  X_n - \mathbb{E}\left(X_1+X_2 + \dots  X_n\right)}{n} \xrightarrow{P} 0$.
\end{zd}

\begin{zd}
Niech $(A_n)$ będą niezależnymi zdarzeniami losowymi i niech $p_n = \mathbb{P}(A_n)$. Wykaż, że wtedy $\frac{\pmb{1}_{A_1} + \pmb{1}_{A_2} + \dots + \pmb{1}_{A_n}}{n} - \frac{p_1 + p_2 + \dots p_n}{n} \xrightarrow{P} 0$
\end{zd}

\begin{zd}
Niech $(X_n)$ będą niezależnymi zmiennymi losowymi takimi, że $\mathbb{P}(X_n=1)=\mathbb{P}(X_n=-1) = p_n$ i $\mathbb{P}(X_n=0)=1-2p_n$. Znajdź warunek konieczny i wystarczający, by ciąg $(X_n)$ spełniał MPWL.
\end{zd}

\end{document}
