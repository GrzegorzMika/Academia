\documentclass{article}
%\usepackage{polski}
%\usepackage[polish]{babel}
\usepackage{amsfonts}
\usepackage{indentfirst}
\usepackage[utf8]{inputenc}
\usepackage{amsthm}
\usepackage{multirow}
\usepackage{amsmath}
\newtheorem{tw}{Twierdzenie}
\newtheorem{df}{Definicja}
\newtheorem{zd}{Zadanie}
\newtheorem{zdt}[zd]{Zadanie*}
\title{Nowe zadania}
\usepackage{Sweave}
\begin{document}
\Sconcordance{concordance:kolokwium.tex:kolokwium.Rnw:%
1 14 1 1 0 48 1}

\noindent Zadanie 1. 
\begin{itemize}
\item Zbadaj zbieżność szeregu:
\begin{itemize}
\item $\sum_{n=1}^{\infty}\frac{\sqrt{n^2 + \sqrt{n}} - \sqrt{n^2-\sqrt{n}}}{\sqrt{n}}$,
\item $\sum_{n=1}^{\infty}\frac{(n!)^n}{n^{n^2}}$,
\item $\sum_{n=1}^{\infty}(-1)^n\frac{\sqrt{n}}{n+100}.$
\end{itemize}
\item Zbadaj zbieżność punktową i jednostajną szeregu $\sum_{n=1}^{\infty}\frac{1}{2^{n-1}\sqrt{1+nx}}$ dla $x \geq 0$.
\end{itemize}
Zadanie 3.
\begin{itemize}
\item Funkcję $f(x)  = \pi -x $ określoną na przedziale $(0, \pi)$ rozwinąć w szereg Fouriera samych sinusów.
\end{itemize}

\noindent Zadanie 1. 
\begin{itemize}
\item Zbadaj zbieżność szeregu:
\begin{itemize}
\item $\sum_{n=1}^{\infty}\frac{(n!)^2}{2^{n^2}}$,
\item $\sum_{n=1}^{\infty}\frac{e^{1/n}}{n}$
\item $\sum_{n=1}^{\infty}\frac{(-1)^n}{n+\sqrt{n}}.$
\end{itemize}
\item Zbadaj zbieżność punktową i jednostajną szeregu $\sum_{n=1}^{\infty}\frac{\ln (1+nx)}{nx^n}$ dla $x \geq 2$.
\end{itemize}
Zadanie 3.
\begin{itemize}
\item Funkcję $f(x)  = \sin x $ określoną na przedziale $[0, \pi]$ rozwinąć w szereg Fouriera samych cosinusów.
\end{itemize}


\noindent Zadanie 1. 
\begin{itemize}
\item Zbadaj zbieżność szeregu:
\begin{itemize}
\item $\sum_{n=1}^{\infty}\frac{1}{\ln^n(n+1)}$,
\item $\sum_{n=1}^{\infty}\frac{1}{\sqrt[4]{n(n+1)(n+2)(n+3)}}$,
\item $\sum_{n=1}^{\infty}(-1)^n\frac{n+2}{n^2+3}.$
\end{itemize}
\item Zbadaj zbieżność punktową i jednostajną szeregu $\sum_{n=1}^{\infty}\frac{e^{-n^2x^2}}{n^2}$ dla $x \in \mathbb{R}$.
\end{itemize}
Zadanie 3.
\begin{itemize}
\item Funkcję $f(x)  = x^2 $ określoną na przedziale $(0, \pi)$ rozwinąć w szereg Fouriera samych sinusów.
\end{itemize}

\end{document}
